\chapter{Aufgabe 3}

\section{Teil 1}

\textit{Ein UHF-RFID-Reader kann nicht nur einzelne, sondern viele Transponder gleichzeitig erfassen (Pulkerfassung, engl. Bulk Reading). Wie
funktioniert das - es müssten dazu die einzelnen Transponder voneinander unterschieden werden?}\\

\noindent
Das \textit{Bulk Reading} sieht sich u.a. zwei Herausforderungen gegenüber:

\begin{enumerate}
    \itemsep0.5em
    \item Zuverlässiges \textit{Lesen} von RFID-Tags in Bereichen mit einer hohen Anzahl gleichzeitig antwortender Tags\footnote{
    siehe hierzu auch Teilaufgaben~\ref{sec:2_4}.
    }
    \item Zuverlässiges \textit{Identifizieren} der Signale zur eindeutigen Zuordnung zu den einzelnen Tags
\end{enumerate}

\noindent
Damit dies funktioniert, kann der \textbf{RSSI} (\textit{Radio Signal Strength Indicator}) verwendet werden, der die gemessene Feldstärkewerte beim Lesen meldet (vgl.~\cite[136]{ES5}) und so dem Leser Auskunft über ungefähre Position und Abstand des Transponders liefert.\\
\textit{Finkenzeller} stellt in~\cite[194]{Fin10} verschiedene Verfahren vor, um das Problem von Signalkollisionen in den Griff zu bekommen, darunter \textit{SDMA} (\textit{Space Division Multiple Access}), bei der die Reichweite einzelner Leser verringert wird, aber mehrere Leser in einem Feld angeordnet sind, um einen bestimmten Bereich zu erfassen - diese Felder werden der Reihe aktiviert/deaktiviert, um Signale, die in den entsprechenden Bereichen vorkommen, zu messen.\\
Protokolle, die auf \textit{TDMA} (\textit{Time Domain Multiple Access}) basieren, spezifizieren u.a. das ``Stummschalten`` des Transponders, sobald der Leser ihr Signal erkannt hat.\\
Ein Verfahren, das auf \textit{TDMA} aufbaut, ist das \textit{Slotted ALOHA}-Verfahren, bei denen der Leser Zeitbereiche (``time slots``) vorgibt, in denen er Signale erwartet.
Die Transponder wählen daraufhin \textit{zufällig} einen Slot aus, in dem sie ihr Signal senden, um mögliche Kollisionen durch gleichzeitiges Senden zu vermeiden:

\blockquote[{\cite[201]{Fin10}}]{
    ``In this procedure, transponders may only begin to transmit data
    packets at defined, synchronous points in time (slots). The synchronisation of all transponders
    necessary for this must be controlled by the reader. This is therefore a stochastic, interrogator-driven
    TDMA anticollision procedure.``
}

