\chapter{Aufgabe 2}

\section{Teil 1)}

\textit{Drücken Sie den Abstand $1$m bei $13.56$ MHz in der entsprechenden Wellenlänge
aus.}\\

\noindent
Die Wellenlänge $\lambda$ ist der Quotient aus Lichtgeschwindigkeit $c$ ($\approx 3 \cdot 10^8 m/s$) und der Frequenz $f$ (\cite[109]{ES5}):

\begin{equation}\label{eq:lambda}
    \lambda = \frac{c}{\frac{1}{T}} = \frac{c}{f}
\end{equation}

\vspace{2mm}

\noindent
Setzen wir die gegebenen Werte in~\ref{eq:lambda} ein, erhalten wir:

\begin{equation}\notag
    \begin{alignat}{3}
        \lambda &= \frac{3\cdot 10^8 \frac{m}{s}}{13.56 \text{MHz}} \\[1em]
        &= \frac{3\cdot 10^8 \frac{m}{s}}{13.56 \cdot 10^6 s^{-1}} \\[1em]
        &= \frac{3\cdot 10^8 m \cdot s^{-1} }{13.56 \cdot 10^6 s^{-1}} \\[1em]
        &= \frac{3\cdot 10^8 m}{13.56 \cdot 10^6} \\[1em]
        &= \frac{3\cdot 10^2 m}{13.56} \\[1em]
        &= \frac{300m}{13.56} \\[1em]
        &\approx 22.12 m
    \end{alignat}
\end{equation}

\vspace{2mm}

\noindent
Insgesamt ergibt sich damit  eine Wellenlänge von ungefähr

\begin{equation}\notag
\lambda \approx 22.12m
\end{equation}


\noindent
\textit{Drücken Sie den Abstand 1 m bei 865 MHz in der entsprechenden Wellenlänge aus.}\\

\noindent
Einsetzen der Werte in die o.a. Gleichung liefert eine Wellenlänge von ungefähr

\begin{equation}\notag
\lambda \approx 35\ \text{cm}
\end{equation}
