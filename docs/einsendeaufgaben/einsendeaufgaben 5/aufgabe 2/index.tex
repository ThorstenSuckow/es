\chapter{Aufgabe 2}

\section{Teil 1)}\label{sec:1_1}

\textit{Drücken Sie den Abstand $1$m bei $13.56$ MHz in der entsprechenden Wellenlänge
aus.}\\

\noindent
Die Wellenlänge $\lambda$ ist der Quotient aus Lichtgeschwindigkeit $c$ ($\approx 3 \cdot 10^8 m/s$) und der Frequenz $f$ (\cite[109]{ES5}):

\begin{equation}\label{eq:lambda}
    \lambda = \frac{c}{\frac{1}{T}} = \frac{c}{f}
\end{equation}

\vspace{2mm}

\noindent
Setzen wir die gegebenen Werte in~\ref{eq:lambda} ein, erhalten wir:

\begin{equation}\notag
    \begin{alignat}{3}
        \lambda &= \frac{3\cdot 10^8 \frac{m}{s}}{13.56 \text{MHz}} \\[1em]
        &= \frac{3\cdot 10^8 \frac{m}{s}}{13.56 \cdot 10^6 s^{-1}} \\[1em]
        &= \frac{3\cdot 10^8 m \cdot s^{-1} }{13.56 \cdot 10^6 s^{-1}} \\[1em]
        &= \frac{3\cdot 10^8 m}{13.56 \cdot 10^6} \\[1em]
        &= \frac{3\cdot 10^2 m}{13.56} \\[1em]
        &= \frac{300m}{13.56} \\[1em]
        &\approx 22.12 m
    \end{alignat}
\end{equation}

\vspace{2mm}

\noindent
Insgesamt ergibt sich damit  eine Wellenlänge von ungefähr

\begin{equation}\notag
\lambda \approx 22.12m
\end{equation}


\noindent
\textit{Drücken Sie den Abstand 1 m bei 865 MHz in der entsprechenden Wellenlänge aus.}\\

\noindent
Einsetzen der Werte in die o.a. Gleichung liefert eine Wellenlänge von ungefähr

\begin{equation}\notag
\lambda \approx 35\ \text{cm}
\end{equation}


\section{Teil 2)}

\textit{NFC - Near Field Communication: Worum handelt sich dabei? Hat das was mit RFID zu tun?}\\


\noindent
Bei \textit{NFC} handelt es sich um ein Kommunikationsprotokoll zur kontaktlosen Datenübertragung auf einer Frequenz von $13.56$ MHz und wird u.a. in \textbf{ISO/IEC 18092}\footnote{
    \url{https://www.iso.org/standard/82095.html}, abgerufen 18.04.2025
} spezifiziert.\\

\noindent
Im Allgemeinen werden Geräte, die auf dieser Frequenz arbeiten, \textbf{HF-Transponder}\footnote{HF = \underline{H}igh \underline{F}requency} genannt, wobei der Abstand zu einem kompatiblen Lesegerät maximal $80$ cm betragen darf\footnote{vgl.~\cite[137]{ES5}}.\\

\noindent
NFC stellt nun einen spezialisierten Anwendungsfall von RFID dar: Die Spezifikation erlaubt eine \textit{bidirektionale Kommunikation} zwischen NFC-Geräten (vgl.~\cite[375 f.]{Fin10}), wobei der typische Kommunikationsabstand von \textit{Finkenzeller} mit $20$ cm angegeben wird (vgl.~\cite[57]{Fin10}).

\section{Teil 3)}

\textit{Abbildung 4 auf der Seite 2 zeigt einen UHF-Transponder. Bestimmen
Sie den Antennentyp.}\\

\noindent
Bei dem in der Abbildung dargestellten RFID-Chip handelt es sich um einen Chip mit einem Durchmesser von $2-3$ mm, der mittig auf einem Transponder angebracht ist: Diese Bauform ist typisch für eine \textbf{$\lambda/2$-Dipolantenne}.\\

\noindent
UFH-Chips haben in Europa einen Frequenzbereich von $865-868$ MHz (vgl.~\cite[165]{Fin10}): Bei einer Wellenlänge $\lambda \approx 35$cm (siehe Aufgabenteil~\ref{sec:1_1}) entspricht dies einer Antennenlänge von $\approx 17$cm für einen $\lambda/2$-Dipol.\\

\noindent
In der Abbildung ist eine Länge von $\approx 11$cm für den Transponder gezeigt - durch die angedeutet \textit{Faltung} von links $\approx 3$cm und rechts $\approx 3$cm erhalten wir in Summe $\approx 17$cm.
Dies entspricht der halben Wellenlänge $\lambda / 2$ bei einer Frequenz von $865$ MHz und passt somit zu einem typischen $\lambda/2$-Dipol.
