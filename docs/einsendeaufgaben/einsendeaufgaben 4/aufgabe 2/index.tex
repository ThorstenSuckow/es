\chapter{Aufgabe 2}

\section{Teil 1}

\textit{Die serielle Schnittstelle Ihres PC-Systems basiert auf einem besonderen Chip, der allerdings vom Linuxkernel nicht unterstützt wird. Sie
schreiben deshalb einen neuen Treiber für die serielle Schnittstelle. Als
Vorlage dient Ihnen der (Standard-)Linuxtreiber, in dem Sie die hardwarerelevanten Teile an Ihren Chip anpassen.
Müssen Sie Ihrem Kunden den Quellcode Ihres Treibers mitliefern?}\\

\noindent
Sofern ``Vorlage`` nicht als ``intelektueller Ansporn`` verstanden wird, sondern als Vorlage, die auch Bestandteile des Linuxtreibers beinhaltet, muss der Quellcode der \textit{GNU GPL} mitgeliefert bzw. auf andere (öffentlich zugängliche) Art und Weise zur verfügung gestellt werden.\\
\textit{Bollenbacher} weist in diesem Zusammenhang auf eine Grauzone hin:

\blockquote[{\cite[184]{ES4}}]{
    ``Der Umgang mit dem Systemtreiber liegt in einer Grauzone. Als Bestandteil des Linuxkernels, der unter der GPL steht, muss er auch unter
    der GPL stehen. [\ldots] Es gibt eine Reihe von Herstellern, die die Systemtreiber ihrer Produkte nicht in Quellform, sondern in übersetzter
    (d.h. als Binärmodul) Form liefern. Dies wird zur Zeit noch geduldet.``

}

\section{Teil 2}

\textit{In einem Internetforum beschreiben Sie unter anderem, wie Sie für Ihren eigenen PC aus einem Standardtreiber (Linux) einen superschnellen
Treiber für beispielsweise Harddisks entwickelt haben.
Einige Personen, die auch an diesem Forum teilnehmen, fordern Sie
dazu auf, den Treiber offenzulegen.
Müssen Sie dem nachkommen?}\\

\noindent
Nein, da der Treiber für den privaten Gebrauch und nicht für die Weitergabe bestimmt ist, muss er nicht veröffentlicht werden.\\
Sollte aus dem privaten Projekt allerdings ein öffentliches/kommerzielles Projekt werden, muss unter dem Copyleft-Hinweis der Quellcode der öffentlichkeit zur Verfügung gestellt werden.

\section{Teil 3}

\textit{Sie übersetzen Ihr C-Programm mit dem unter GPL stehenden GNUC-Compiler.
Müssen Sie deshalb Ihrem Kunden auch Ihre C-Quellen mitliefern?}\\

\noindent
Nein, da die GNUC-Standardbibliotheken unter der LGPL veröffentlich sind.\\
Diese ist weniger viral/restriktiv als die GPL.


\section{Teil 4}

\textit{Sie setzen eine x86-CPU mit 32 Bit ein. Wie viele Bytes können Sie
    damit direkt adressieren?
    Wenn Sie auf dieser CPU ein Linux betreiben, wie groß darf dann Ihr
    Anwenderprogramm maximal sein?
}\\

\noindent
Mit 32 Bit lassen $2^{32}$ Speicheradressen adressieren - mit einer Wortbreite von 8Bit für eine Adresse werden also ebensoviele Bytes direkt adressiert.\\
Das Programm darf maximal $3$GB groß sein, da von den $2^{32}$ Bit insg. 1 GB für den Kernel reserviert sind (vgl.~\cite[3 f.]{ES4}).


\section {Teil 5}

\textit{Der GNU-C-Compiler kennt die folgenden Optionen:
\begin{itemize}
    \item \texttt{-l}
    \item \texttt{-Wformat}
\end{itemize}
    Was bedeuten diese Optionen?
}\\

\noindent
\texttt{-Wformat}\\
Ein Aufruf von \texttt{man gcc} liefert für \texttt{-l}\footnote{
etwas übersichtlicher auch unter \url{https://man7.org/linux/man-pages/man1/gcc.1.html} (abgerufen 10.04.2025)
}:

\blockquote[]{
``-l library\\
Search the library named library when linking.  (The second
    alternative with the library as a separate argument is only
    for POSIX compliance and is not recommended.)``
}\\

\noindent
Damit ist es also möglich, für den Kompilierungsvorgang - statisch oder dynamisch - die zu linkenden Bibliotheken anzugeben, wobei:

\blockquote[]{
    ``If
    both static and shared libraries are found, the linker gives
    preference to linking with the shared library unless the
    -static option is used.``
}

\noindent
\texttt{-Wformat}\\

\noindent
Laut \texttt{man gcc} ist ``-Wformat is equivalent to -Wformat=1``.\\

\noindent
\texttt{-Wformat=n}  überprüft Aufrufe zu \texttt{printf} und \texttt{scanf}, und ob die Typen der übergebenen Argumente zu den Platzhaltern im Format-String passend sind.

\noindent
Das unterschiedliche Verhalten kann anhand des folgendes Programms nachvollzogen werden:

\begin{minted}[mathescape,
    numbersep=5pt,
    gobble=2,
    frame=none,
    framesep=2mm]{java}
    #include <stdio.h>
    int main() {
        printf("%f", 1);
        return 0;
    }
\end{minted}

\noindent
Im Formatstring ist \texttt{\%f} angegeben, was in C als Platzhalter für einen Fliesskommatypen \textit{float}/\textit{double} vorgesehen ist (\cite[18, 244]{KR88}).\\

\noindent
Kompilieren ohne Angabe von \texttt{Wformat} oder unter Angabe von \texttt{Wformat=1} liefert:\\

\begin{lstlisting}[
    basicstyle=\ttfamily\footnotesize,
    frame=none,
    keepspaces=true,
    showspaces=false
]
./HelloWorld.c:5:27: warning: format ‘%f’ expects argument of type ‘double’,
                    but argument 2 has type ‘int’ [-Wformat=]
5 |     printf("Hello World! %f", 1);
  |                          ~^   ~
  |                           |   |
  |                           |   int
  |                           double
  |                          %d
\end{lstlisting}

\noindent
während \texttt{gcc HelloWorld.c -Wformat=0} keine Warnung des Compilers erzeugt