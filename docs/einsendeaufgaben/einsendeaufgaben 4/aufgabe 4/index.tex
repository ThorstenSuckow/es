\chapter{Aufgabe 4}

\section{Teil 1}

\textit{Was ist eine \texttt{Proc}-Datei?}\\

\noindent
Die Architektur von Unix ist sehr dateiorientiert ausgelegt.
Es finden sich nicht nur Geräteinformationen im Verzeichnissystem (\texttt{/dev}), sondern es werden auch Prozessinformationen im Verzeichnissystem abgelegt.
\textit{Pseudodateien} findet man auch unter \texttt{/proc}: So hält \texttt{/proc/[PID]/maps} Informationen zu den Prozessen mit den jeweiligen PIDs und ihren Speicherzuordnungen vor (vgl.~\cite[103]{ES4}]).\\

\noindent
Als \texttt{Proc}-Datei bezeichnet man unter Unix also ganz allgemein (Pseudo-)Dateien, die \textbf{System-} und \textbf{Prozessinformationen} zur Verfügung stellen.

\section{Teil 2}

\textit{Gibt es eine Angabe Interrupt und Cached in der Proc-Datei \texttt{/proc/meminfo}?}\\

\noindent
Ein Aufruf von

\begin{verbatim}
       egrep "^Cached" /proc/meminfo
\end{verbatim}

\noindent
liefert den Treffer


\begin{center}
    \texttt{Cached:          1212384 kB}
\end{center}

\noindent
wohingegen

\begin{verbatim}
       egrep "^Interrupt" /proc/meminfo
\end{verbatim}

\noindent
keine Treffer liefert.
Eine manuelle Inspektion aller Einträge über

\begin{verbatim}
       cat /proc/meminfo | sort | less
\end{verbatim}


\noindent
bestätigt dies:

\begin{lstlisting}[
    basicstyle=\ttfamily\footnotesize,
    frame=single,
    keepspaces=true,
    showspaces=false
]
...
Inactive:         553372 kB
Inactive(anon):    90440 kB
Inactive(file):   462932 kB
KernelStack:        6896 kB
KReclaimable:     179680 kB
Mapped:           217448 kB
...
\end{lstlisting}


\section{Teil 3}

\textit{Welche Wirkung haben die beiden Optionen \texttt{-o} und \texttt{-x} von \texttt{egrep}?}\\

\noindent
\texttt{-o}: liefert nur die durch den regulären Ausdruck gefundene Zeichenkette und nicht die gesamte Zeile, in der der Treffer vorkommt\footnote{
siehe auch Lösungsvorschlag zu Aufgabe 3)
}. \texttt{man egrep} sagt u.a.:

\begin{verbatim}
-o, --only-matching
              Print  only  the matched (non-empty) parts of a matching
              line, with each such part on a separate output line.
\end{verbatim}

\\

\noindent
\texttt{-x}: weist \texttt{egrep} an, den Suchausdruck auf die gesamte Zeile anzuwenden:\\

\begin{verbatim}
-x, --line-regexp
              Select only those matches that exactly match the
              whole line. For a regular expression pattern, this
              is like parenthesizing the pattern and then
              surrounding it with ^ and $.
\end{verbatim}

\noindent
\code{^} und \code{$} werden als \textit{Anker} bezeichnet und kennzeichnen Zeilenbeginn und Zeilenende.
Mit ihrer Hilfe kann man den Suchausdruck weiter einschränken:
So hatten wir bei der Suche nach \texttt{Cached} in dem vorherigen Aufgabenteil bspw. den Suchausdruck

\begin{verbatim}
    ^Cached
\end{verbatim}

\noindent
verwendet, um Treffer wie \texttt{SwapCached} in der Ergebnisliste auszuschließen.
Allerdings können wir an dieser Stelle nicht einfach mit dem Anker \code{$} arbeiten, da ein Eintrag in \texttt{/proc/meminfo} mit Informationen über den verbrauchten Speicher endet (Angabe in kB). \\

\noindent
So liefert

\begin{verbatim}
    egrep "^Cached$" /proc/meminfo
\end{verbatim}

\noindent
bzw.


\begin{verbatim}
egrep -x "Cached" /proc/meminfo
\end{verbatim}


\noindent
keine Treffer - wir müssen an dieser Stelle

\begin{verbatim}
egrep "^Cached.*kB$" /proc/meminfo
\end{verbatim}

\noindent
bzw.

\begin{verbatim}
egrep -x "Cached.*kB" /proc/meminfo
\end{verbatim}

\noindent
verwenden.

\section{Teil 4}

\textit{Schreiben Sie ein Skript \texttt{Procfilt}, das periodisch die Werte von \texttt{MemTotal}
und \texttt{MemFree} der Proc-Datei \texttt{/proc/meminfo} überprüft und die Differenz der beiden Werte ausrechnet
und auf dem Terminal ausgibt. Es reicht aus, wenn die Überprüfung alle 5 Sekunden erfolgt.}\\



\noindent
Die Lösung ist in Listing~\ref{lst:procfilt} angegeben}.
Wir erlauben die Angabe des gewünschten Intervalls als Argument, es findet allerdings nur ein einfaches Casting auf einen Ganzzahlwert statt, keine Überprüfung auf Format des angegebenen Arguments.\\

\noindent
Das Auslesen der Werte aus \texttt{/proc/meminfo} wird über ein gepiptes \texttt{egrep} realisiert - im ersten Schritt wird die Zeile ausgelesen, der zweite Schritt liefert den kB-Wert:

\begin{verbatim}
egrep -o  "^MemTotal(.*)kB$" /proc/meminfo | egrep -o "[[:digit:]]{0,}
\end{verbatim}

\noindent
Wir nutzen den Wagenrücklauf\texttt{\textbackslash r}, um die Ausgabe zeilenweise zu überschreiben und somit etwas übersichtlicher zu halten.

\begin{listing}[H]
\begin{minted}[linenos,fontsize=\small]{bash}
#! /bin/bash
echo "Usage: procfilt.sh [n] "
echo "       n = Interval in seconds, defaults to 5"

default="5"
arg1=${1}
interval=$((0 + arg1))

if [ ${interval} -le 0 ]
    then interval=${default}
fi

while true
    do
        memtotal=$(egrep -o  "^MemTotal(.*)kB$" /proc/meminfo | egrep -o "[[:digit:]]{0,}")
        memfree=$(egrep -o  "^MemFree(.*)kB$" /proc/meminfo | egrep -o "[[:digit:]]{0,}")
        printf  "\r $((memtotal - memfree))"
        sleep ${interval}
    done
\end{minted}
\caption{\texttt{procfilt.sh} zur Ausgabe der Differenz von \texttt{MemTotal} und \texttt{MemFree} aus \texttt{/proc/meminfo}.}
\label{lst:procfilt}
\end{listing}