\chapter{Aufgabe 5}

\section{Teil 1}
\textit{Was bewirkt \texttt{sudo date -s 16:30:20}?}\\

\noindent
Die Option \texttt{-s} steht für \textit{set} und bewirkt die Aktualisierung des Systemdatums mit dem angegebenen Wert:

\begin{verbatim}
   -s, --set=STRING
      set time described by STRING
\end{verbatim}

\noindent
Das Setzen des Systemdatums ist dem \textit{root}-User vorbehalten, weswegen das Kommando unter Anwendung  von \texttt{sudo} durchgeführt werden muss, sofern der \texttt{root}-User nicht bereits für die geg. Sitzung angemeldet ist.


\section{Teil 2}

\textit{Schreiben Sie ein Skript \texttt{Sirene\_Ein}, das die Sirene später im Betrieb
hardwaremäßig einschalten wird. Für Ihre Zwecke reicht es, wenn dieses
Skript lediglich die Meldung ``Sirene eingeschaltet`` ausgibt und sich
dann beendet.}\\

\noindent
Das Skript ist in Listing~\ref{lst:sireneein} angegeben.

\begin{listing}[H]
    \begin{minted}[linenos,fontsize=\small]{bash}
#! /bin/bash
echo "Sirene eingeschaltet"
exit 0
    \end{minted}
    \caption{\texttt{Sirene\_Ein.sh}}
    \label{lst:sireneein}
\end{listing}

\section{Teil 3}

\textit{Schreiben Sie ein Skript \texttt{Sirene\_Aus}, das die Sirene später im Betrieb
hardwaremäßig ausschalten wird. Für Ihre Zwecke reicht es, wenn dieses
Skript lediglich die Meldung ``Sirene ausgeschaltet`` ausgibt und sich
dann beendet.}\\

\noindent
Das Skript ist in Listing~\ref{lst:sireneaus} angegeben.

\begin{listing}[H]
    \begin{minted}[linenos,fontsize=\small]{bash}
#! /bin/bash
echo "Sirene ausgeschaltet"
exit 0
    \end{minted}
    \caption{\texttt{Sirene\_Aus.sh}}
    \label{lst:sireneaus}
\end{listing}


\section{Teil 4}

\textit{Schreiben Sie ein Skript \texttt{Sirene}, das zu jedem Schichtende die Sirene
ca. eine Minute lang ertönen lässt. Das Skript wird beim Einschalten
des Rechners gestartet und arbeitet im Hintergrund, bis der Rechner
wieder ausgeschaltet wird. Wichtig ist, dass das Schichtende möglichst
genau erkannt wird. Überlegen Sie sich, wie erreicht werden kann, dass
das Skript nicht unnötig viel CPU-Zeit in Anspruch nimmt.}\\

\noindent
Das Skript ist in Listing~\ref{lst:sirene} angegeben.\\

\noindent
Um möglichst wenig CPU-Zeit zu verbrauchen, aber noch verhältnismäßig genau zu arbeiten, wird überprüft, wieviel Sekunden zwischen Schichtende und Zeitmessung vergangen sind - die Differenz wird dann beim ersten Warten auf die nächste volle Minute abgezogen. \\

\noindent
Ein Test mit Schichtende \texttt{14:00} und \texttt{14:02} und starten des Skripts um \texttt{14:00:25} liefert folgende Ausgabe:

\begin{verbatim}
Suche nach laufender Sirenensteuerung...
Killing previous Sirene.sh 85848
Sirenensteuerung läuft (86807).
14:00:25 : 25 Sekunden werden für
           den nächsten Tick abgezogen.
Sirene eingeschaltet
14:00:25 Sirene eingeschaltet.
Sirene ausgeschaltet
14:01:25 Sirene ausgeschaltet.
Sirene eingeschaltet
14:02:00 Sirene eingeschaltet.
Sirene ausgeschaltet
14:03:00 Sirene ausgeschaltet.
\end{verbatim}

\begin{listing}[H]
    \begin{minted}[linenos,fontsize=\small]{bash}
#! /bin/bash

# after DURATION seconds, a running sirene will be switched off
DURATION=60

# wait for WAIT seconds in between time measurements
WAIT=60

# compute the seconds that have already passed once
# we branch into end of shift. if WAIT-OVERFLOW is less than
# zero, we might have encountered leap seconds,
# which will not be considered in further delta calculations
OVERFLOW=0

# remove previous processes
echo "Suche nach laufender Sirenensteuerung..."
prevproc=$(ps hS -o pid -C Sirene.sh)
for i in $prevproc
    do
        if [ ${i} -ne $$ ]
            then
                echo "Killing previous Sirene.sh ${i}"
                kill ${i}
        fi
    done


echo "Sirenensteuerung läuft (${$})."

while true
    do

        time=$(date "+%H:%M")

        case "${time}" in
            "06:00" | "14:00" | "22:00")
                OVERFLOW=$(date "+%-S")

                if [ ${OVERFLOW} -gt 0 ]
                    then
                        echo "$(date "+%H:%M:%S") : ${OVERFLOW} Sekunden werden für"
                        echo "           den nächsten Tick abgezogen."
                fi

                ./Sirene_Ein.sh
                echo "$(date '+%H:%M:%S') Sirene eingeschaltet."
                sleep ${DURATION}
                ./Sirene_Aus.sh
                echo "$(date '+%H:%M:%S') Sirene ausgeschaltet."
            ;;
        esac

        if [ $((WAIT - OVERFLOW)) -lt 0 ]
            then
                sleep 0
            else
                sleep $((WAIT - OVERFLOW))
            fi
        OVERFLOW=0
    done
    \end{minted}
    \caption{\texttt{Sirene.sh}}
    \label{lst:sirene}
\end{listing}
