\chapter{Aufgabe 4}

\section{Teil 1}


\textit{Die Zahl $170_{(10)}$ soll in eine Dualzahl gewandelt werden.}\\

\begin{equation}
\begin{alignat}{3}
    170_{(10)}  \implies 170 &  : 2  = 85 && |  0\\
      85 & : 2 = 42 &&| 1\\
      42 & : 2  = 21 &&| 0\\
      21 & : 2  = 10 &&| 1\\
      10 & : 2  = 5 && | 0\\
      5 & : 2  = 2 && | 1\\
      2 & : 2  = 1 && | 0\\
      1 & : 2  = 0 && | 1 \\
    & && \implies (1 0 1 0 \ 1 0 1 0)_{2}
\end{alignat}
\end{equation}

\section{Teil 2}


\textit{Die Zahl $170_{(10)}$ soll in eine Hexadezimalzahl gewandelt werden.}\\

\begin{equation}
    \begin{alignat}{3}
        170_{(10)}  \implies 170 &  : 16 = 10 && | 10\\
        10 & : 16  = 0 && | 10
        & && \implies (AA)_{16}
    \end{alignat}
\end{equation}


\section{Teil 3}


\textit{Die Zahl $170_{(10)}$ soll in eine Oktalzahl gewandelt werden.}\\

\begin{equation}
    \begin{alignat}{3}
        170_{(10)}  \implies 170 &  : 8  = 21 && | 2\\
        21 & : 8  = 2 && | 5 \\
        2 & : 8  = 0 && | 2 \\
        & && \implies (252)_{8}
    \end{alignat}
\end{equation}

\section{Teil 4}


\textit{Die Zahl $170_{(16)}$ soll in eine Oktalzahl gewandelt werden.}\\

\begin{equation}
    \begin{alignat}{3}
        170_{(16)}  \implies &  (0001 \ 0111 \ 0000)_{(2)} && \text{(nibbles berechnen)} \\
                    \implies &  000 \ 101 \ 110 \ 000 && \text{(zu 3er-Gruppen zusammenfassen)} \\
                    \implies &  560_{(8)} && \text{(3er-Gruppen von Binär nach Oktal)}
    \end{alignat}
\end{equation}

\section{Teil 5}


\textit{Die Zahl $170_{(8)}$ soll in eine Dualzahl gewandelt werden.}\\

\noindent
Auch bei der folgenden Umrechnung werden die jeweiligen Stellen in Gruppen von 3er-Bits umgerechnet, dann als Binärfolge zusammengefasst\footnote{
    Die führende $0$ ist beibehalten, damit das rechte Nibble nicht so alleine ist.
}.\\


\begin{equation}
    \begin{alignat}{3}
        170_{(8)}  \implies 001 \ 111 \ 000 \implies &  (0111\ 1000)_{(2)}
    \end{alignat}
\end{equation}

\section{Teil 6}


\textit{Wie bewerten Sie die folgende Aussage: $4 \overset{\wedge}{=} 100$}\\

\noindent
Unter der Annahme, dass $\overset{\wedge}{=}$ gelesen werden kann als ``ist definiert durch``\footnote{
    \url{https://en.wikipedia.org/wiki/Glossary_of_mathematical_symbols}, abgerufen 25.03.2025
} (auch $\coloneq$), kann die Aussage gelesen werden als

\begin{equation}\notag
    4_{(10)} \coloneq 100_{(2)}
\end{equation}

\noindent
Der Nachweis ist entsprechend trivial, wir dürfen die Aussage also im Grunde genommen als korrekt bewerten, auch, wenn sie formal etwas schwächelt.\\

\noindent
\textit{Wie bewerten Sie die folgende Aussage: $14 \overset{\wedge}{=} 24$}\\

\noindent
Wir folgen der o.a. Argumentation und rutschen in das hexadezimale Zahlensystem, denn es gilt:

\begin{equation}\notag
14_{(16)} = (0001 \ 0100)_{(2)} = (00 \ 010 \ 100)_{(2)} = 24_{(8)}
\end{equation}

\noindent
Im Zwischenschritt haben wir uns die Freiheit genommen, die Bits ein wenig zu verschieben, um den Übergang in das Oktalsystem übersichtlicher zu gestalten.