\chapter{Aufgabe 1}

\section{Teil 1)}

\textit{Warum ist es wichtig zu wissen, an welcher Speicherstelle das Programm startet?}\\

\noindent
Würde die Information fehlen, an welcher Stelle ein Programm startet,  würde u.a. Speicher nicht richtig initialisiert oder zurückgesetzt: So zeigen \textit{Bollenbacher und Liell} in~\cite{BL21}[\textbf{Tabelle 19}, 86] ein Programm zur Addition zweier Zahlen auf dem M6800\footnote{
\url{https://en.wikipedia.org/wiki/Motorola_6800}, abgerufen 22.03.2025
}, das mit der Instruktion \texttt{CLR A}\footnote{
s. a. \url{http://www.8bit-era.cz/6800.html#CLR}, abgerufen 22.03.2025
} beginnt, und damit den Inhalt des  Akkumulators\footnote{
Register zur Aufnahme von Operanden bzw. Ergebnissen einer Rechenoperation, vgl.~\cite{Fri21}[233])
} löscht.
Würde in diesem Beispiel die Startadresse des Programms nicht bekannt sein, würde - sofern überhaupt eine (sinnvolle) Rechenoperation ausgeführt wird - unter hoher Wahrscheinlichkeit mit Artefakten vorhergehender Rechenoperationen gearbeitet bzw. das Programm in einem fremden Adressraum operieren.\\

\noindent
Insgesamt darf man wohl behaupten, dass ohne Kenntnis der Startadresse eines Programm ein (deterministischer) \textbf{Rechenbetrieb nicht möglich} wäre, zumal darauffolgende anzusteuernde Operationen über den Befehlszähler auf den nächsten auszuführenden Befehl zeigen\footnote{
Ein \textbf{Befehlszähler} als Teil eines \textbf{Leitwerks}\footnote{
    hier insb. bei der Von-Neumann-Rechnerarchitektur
} ist ein Register, der die Adressen des als nächstes auszuführenden Maschinen-Befehls enthält (vgl.~\cite{Fri21}[234]).
}, und sich daraus ja gerade die auszuführenden Rechneoperationen ergeben.\\

\noindent
In diesem Zusammenhang darf man die Startadresse wohl wie den Startzustand eines Turingautomaten verstehen, der entsprechend einer geg. Zustandsüberführungsfunktion $\delta$ (dem \textit{Turingprogramm}, vgl.~\cite{VW16h}[269 f.]) - eine Eingabe berarbeitet.

\section{Teil 2}

Der Lösungsvorschlag ist in Tabelle~\ref{tab:speicherinstruktionen} angegeben.

\begin{table}[h!]
    \setlength{\tabcolsep}{0.5em}
    \def\arraystretch{1.5}
    \centering
    \begin{tabular}{|c|c|l|l|}
        \hline
        \textbf{Zeile} & \textbf{Speicheradresse} & \multicolumn{2}{c|}{\textbf{Instruktion}} \\
        \cline{3-4}
        &                          & \textbf{Hex-Code} & \textbf{Mnemonics (Kürzel)} \\
        \hline
        $1$ & $0000$ & \texttt{4f} & \texttt{start CLR A}  \\

        $2$ & $0001$ & \texttt{b6 00 14} & \qquad\quad\texttt{LDA A value}  \\

        $3$ & $0004$ & \texttt{b7 00 12} & \qquad\quad\texttt{STA A result1}  \\

        $4$ & $0007$ & \texttt{8b 0f} & \qquad\quad\texttt{ADD A \#15}  \\

        $5$ & $0009$ & \texttt{7e 00 0c} & \qquad\quad\texttt{JMP next}  \\

        $6$ & $000c$ & \texttt{b7 00 13} & \texttt{next STA A result2}  \\

        $7$ & $000f$ & \texttt{7e 00 00} & \qquad\quad\texttt{JMP start}\\
        &&&\\
        $8$ & $0012$ & \texttt{00} & \texttt{result1 byte 00}  \\
        $9$ & $0013$ & \texttt{00} & \texttt{result2 byte 00}  \\
        $10$ & $0014$ & \texttt{40} & \texttt{value byte 40}  \\

        \hline
    \end{tabular}
    \caption{Lösungsvorschlag zu Aufgabe 1.2}
    \label{tab:speicherinstruktionen}
\end{table}
