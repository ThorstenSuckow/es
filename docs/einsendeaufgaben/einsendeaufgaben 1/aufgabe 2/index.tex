\chapter{Aufgabe 2}

\section{Teil 1}

\textit{Geben Sie die Schaltfunktionen von $y_3$, $y_2$ und $y_1$ an.}\\


\noindent

\noindent
Zur Erstellung der Schaltfunktionen für die jeweiligen Ausgänge $y_3$, $y_2$, $y_1$ überführen wir zeilenweise in die DNF.\\
Hierzu bilden wir für die jeweiligen Ausgänge die Minterme\footnote{
    ``für genau eine Variablenbelegung wahr``~\cite[92]{Hof22}
} und verknüpfen sie dann disjunktiv (Ziffern in Klammern beziehen sich auf die Zeilen der in der Aufgabenstellung angegebenen Tabelle).\\

\noindent
Für $y_3$ erhalten wir die DNF wie in~\ref{eq:dnf_y3} angegeben:


\begin{equation}\label{eq:dnf_y3}
\begin{alignat}{3}
    y_3 =\ &(\neg a \ \land \phantom{\neg} b \ \land \phantom{\neg} c \ \land \phantom{\neg} d)\ \lor && \text{(7)} \\
          &(\phantom{\neg} a \ \land \phantom{\neg} b \ \land \phantom{\neg} c \ \land \neg d) && \text{(14)}
\end{alignat}
\end{equation}

\noindent
Für $y_2$ erhalten wir die DNF wie in~\ref{eq:dnf_y2} angegeben:


\begin{equation}\label{eq:dnf_y2}
\begin{alignat}{3}
    y_2 =\ &(\neg a \ \land \phantom{\neg} b \ \land \neg c \ \land \phantom{\neg} d)\ \lor && \text{(5)} \\
    &(\phantom{\neg} a \ \land \neg b \ \land \neg c \ \land \phantom{\neg} d)\ \lor && \text{(9)}  \\
    &(\phantom{\neg} a \ \land \neg b \ \land  \phantom{\neg} c \ \land \neg d) && \text{(10)}
\end{alignat}
\end{equation}

\noindent
Für $y_1$ erhalten wir die DNF wie in~\ref{eq:dnf_y1} angegeben:


\begin{equation}\label{eq:dnf_y1}
\begin{alignat}{3}
    y_1 =\ &(\neg a \ \land \neg b \ \land \neg c \ \land \phantom{\neg} d)\ \lor && \text{(1)}  \\
    &(\neg a \ \land \neg b \ \land  \phantom{\neg} c \ \land \neg d)\ \lor && \text{(2)}  \\
    &(\neg a \ \land  \phantom{\neg} b \ \land \neg  c \ \land \neg d)\ \lor && \text{(4)}  \\
    &(\phantom{\neg} a \ \land  \phantom{\neg} b \ \land \neg  c \ \land \neg d)\ \lor && \text{(8)}  \\
    &(\phantom{\neg} a \ \land  \neg b \ \land \phantom{\neg} c \ \land  \phantom{\neg} d)\ \lor && \text{(11)}  \\
    &(\phantom{\neg} a \ \land   \phantom{\neg} b \ \land \neg c \ \land  \phantom{\neg} d) && \text{(13)}
\end{alignat}
\end{equation}