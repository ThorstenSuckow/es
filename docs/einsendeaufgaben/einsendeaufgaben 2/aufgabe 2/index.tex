\chapter{Aufgabe 2}

\section{a)}

Sei der folgende Programmcode in C gegeben:

\begin{minted}[mathescape,
    numbersep=5pt,
    gobble=2,
    frame=none,
    framesep=2mm]{c}
    int ret, status;
    ret = fork();

    if (ret == 0) {
        printf("K1\n");
        exit(0);
    } else {
        printf("E1\n");
        wait(&status);
        printf("E2\n");
        exit(0);
    }
\end{minted}\\



\begin{itemize}
    \itemsep0.5em
    \item \textbf{F:} Was bedeuten wohl die Buchstaben ``K`` und ``E`` in den Ausgabeanweisungen?
    \item \textbf{A:} \textit{K} und \textit{E} beziehen sich in den Ausgabeanweisungen als Hinweis darauf, welcher von dem Programm ausgelöste Prozess gerade ausgeführt wird: Ist \texttt{ret == 0} (als Rückgabewert von \code{fork()}\footnote{
        \url{https://man7.org/linux/man-pages/man2/fork.2.html}, abgerufen 26.03.2025
    }), dann befindet sich die Ausführung in dem von dem \textbf{E}lternprozess gestarteten \textbf{K}indprozess
    \item[]
    \item \textbf{F:} Was ist der grundlegende Zweck der \texttt{if‐else}‐Anweisung?
    \item \textbf{A:} der Zweck ist es, in einem Programm zwei verschiedene Ausführungszweige für den \textbf{E}lternprozess und den \textbf{K}indprozess zu implementieren.
    Dadurch ist es bspw. auch möglich, in dem Programm eine Routine zur IPC (\textit{Inter-Process-Communication}) zwischen den Prozessen unterzubringen, was \textit{Mandl} in~\cite[180 f.]{Man20g} anhand von \textit{Pipes} demonstriert.
    \item[]
    \item \textbf{F:} Wird ``E2`` immer vor oder immer nach ``K1`` ausgegeben? Begründung!
    \item \textbf{A:} \textit{E2} wird immer \textbf{nach} \textit{K1} ausgegeben, da der Elternprozess vorher den \textbf{Systemaufruf} \code{wait()} durchführt:
    \blockquote[{\url{https://man7.org/linux/man-pages/man2/wait.2.html}, abgerufen 26.03.2025}]{
        ``The wait() system call suspends execution of the calling thread
        until one of its children terminates.``
    }\\
    \noindent
    In dem Programm wird der Methode außerdem noch eine Referenz auf die Speicherstelle übergeben, die den Integer-Wert für die Variable \code{status} vorhält: Sobald der Kindprozess beendet wurde, enthält  \code{status} eine Bitmaske, die verschiedene Informationen über die Ausführung des Kindprozesses beinhaltet (normal abgebrochen, durch Signal abgebrochen usw.)
    \item[]
    \item \textbf{F:} Lässt sich eine ähnliche Aussage für ``E1`` und ``K1`` treffen? Begründung!
    \item \textbf{A:} Es lässt sich für den allgemeinen Fall keine ähnliche Aussage treffen, da an dieser Stelle das Prozess-Scheduling des Betriebssystems greift. Es kann in manchen Fällen sein, das ``K1`` vor ``E1`` ausgegeben wird. Es gilt aber genauso andersrum.\\
    Als Anmerkung: Problematisch ist in solchen Fällen das komplette Auslassen von \code{wait()}\footnote{was durchaus auch durch fehlerhafte Programmierung geschehen kann} - würde bspw. in dem Programm der Aufruf zu \code{wait()} fehlen, ist es durchaus möglich, dass die CPU dem Elternprozess die CPU-Zeit zuweist, dieser durchläuft, sich beendet und im Anschluss der Kindprozess ausgeführt wird - ohne zugehörigen Elternprozess. In diesem Fall verwaisen die Kindprozesse, und werden als Zombi-Prozessen dem ``Ur-Prozess`` zugeordnet (vgl.~\cite[94 f.]{Man20e}])
\end{itemize}


\section{b)}

Sei der folgende Programmcode in C gegeben:

\begin{minted}[mathescape,
    numbersep=5pt,
    gobble=2,
    frame=none,
    framesep=2mm]{c}
    int ret;
    ret = fork();

    if (ret == 0) {
        printf("Meine PID ist %d\n", ret);
        exit(0);
    } else {
        printf("Die PID meines Kinds ist %d\n", ret);
        exit(0);
    }
\end{minted}\\

\noindent
Wir können folgendes feststellen:

\begin{itemize}
    \itemsep0.5em
    \item \code{printf()} Kindprozess: Die Ausgabe der PID des Kindprozesses ist \textbf{nicht korrekt}.
    \code{ret} wird \code{0} zugewiesen, weshalb die Ausgabe nicht die tatsächliche Prozess-ID des Kindprozesses enthält.\\
    An dieser Stelle müsste die Funktion \code{getpid()}\footnote{
\url{https://man7.org/linux/man-pages/man2/getpid.2.html}, abgerufen 27.03.2025
    } verwendet werden.
    \item \code{printf()} Elternprozess: Die printf-Ausgabe des Elternprozess ist \textbf{bzgl. der gewünschten Ausgabe} korrekt.
    Allerdings sollte \code{wait()} verwendet werden, damit der Elternprozess auf die Beendigung des Kindprozesses wartet und der Kindprozess korrekt terminieren kann.
\end{itemize}

\noindent
Der erweiterte Code zur korrekten Ausgabe der Prozess-IDs lautet (als kompilierfähiges C-Programm) wie folgt:

\begin{minted}[mathescape,
    numbersep=5pt,
    gobble=2,
    frame=none,
    framesep=2mm]{c}
    #include <stdlib.h>
    #include <stdio.h>
    #include <unistd.h>
    #include <sys/wait.h>

    void main() {

        int status;
        int ret = fork();

        if (ret == 0) {
            printf("K: Meine PID ist %d\n", getpid());
            printf("   Meine Parent-PID ist %d\n", getppid());
            exit(0);
        } else {
            wait(&status);
            printf("E: Meine PID ist %d\n", getpid());
            printf("   Die PID meines Kindprozesses ist %d\n", ret);
            exit(0);
        }
    }
\end{minted}\\


\section{c)}
Sei der folgende Programmcode in Java gegeben:

\begin{minted}[mathescape,
    numbersep=5pt,
    gobble=2,
    frame=none,
    framesep=2mm]{java}
    class MyThread extends Runnable {

        public void run() {
            System.out.println("Hier ist der Thread.");
            System.out.println(
                "Ich werde nebenläufig zum Thread des Hauptprogramms ausgeführt."
            );
        }

    }
\end{minted}\\

\noindent
Das Hauptprogramm beinhaltet folgende Zeilen:

\begin{minted}[mathescape,
    numbersep=5pt,
    gobble=2,
    frame=none,
    framesep=2mm]{java}
    MyThread thr = new MyThread();
    thr.run();
\end{minted}\\

\noindent
Wir können folgendes feststellen:

\begin{itemize}
    \itemsep0.5em
    \item Die Klasse \code{MyThread} ist falsch implementiert: \code{Runnable} ist eine Schnittstelle.
    Entweder muss \code{MyThread} von \code{java.lang.Thread} erben, oder der Klassenheader muss umgeschrieben werden zu
    \item[] \begin{center}\code{class MyThread implements Runnable}\end{center}
    \item Nach dieser Korrektur ist der gegebene Code-Ausschnitt in der \code{main()} (weiterhin) syntaktisch korrekt, aber \textit{semantisch irreführend}: Durch den Methodenaufruf wird nicht implizit ein neuer Thread erstellt, sondern einfach nur im \textit{main}-Thread\footnote{
    ``the thread that typically calls the application's main method``, \url{https://download.java.net/java/early_access/loom/docs/api/java.base/java/lang/Thread.html}, abgerufen 27.03.2025
    } des Programms die Anweisungen ausgeführt.\\
    Damit die Anweisungen in \code{run()} der Klasse \code{MyThread} tatsächlich in einem nebenläufigen Thread ausgeführt werden, müsste der Code wie folgt umgeschrieben werden\footnote{unter der Annahme, dass mindestens eine aktuelle Java-Version - wie bspw. 21 - verwendet wird}:
\end{itemize}


\begin{minted}[mathescape,
    numbersep=5pt,
    gobble=2,
    frame=none,
    framesep=2mm]{java}
    class MyThread implements Runnable {
       ...
    }

    Thread thr = new Thread(new MyThread());
    thr.start();
\end{minted}\\

\vspace{5mm}

\noindent
Um einen weiteren Thread zu erzeugen, der genau die gleichen Anweisungen ausführt, kann die main-Methode des Programms wie folgt ergänzt werden:

\begin{minted}[mathescape,
    numbersep=5pt,
    gobble=2,
    frame=none,
    framesep=2mm]{java}

    Thread thr = new Thread(new MyThread());
    Thread t2  = new Thread(new MyThread());

    thr.start();
    t2.start();
\end{minted}\\