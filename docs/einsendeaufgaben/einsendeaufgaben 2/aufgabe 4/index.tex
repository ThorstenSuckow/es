\chapter{Aufgabe 4}


\section{a)}
\noindent
Der Lösungsvorschlag ist in Tabelle~\ref{tab:adresszuordnung_a} angegeben.\\
Die reale physische Adresse berechnet sich dabei nach der Formel

\begin{equation}\notag
    A_{\text{real}} = A_{\text{phys}} + (A_{\text{virt}} \ \text{MOD} \ S_{\text{B}})
\end{equation}

\noindent
wobei $S_{\text{B}}$ für die Seitengröße (in Bytes) und $A_{\text{phys}}$ für die Startadresse der zugeordneten Seite im physischen Speicher steht.

\begin{table}[h!]
    \setlength{\tabcolsep}{0.5em}
    \def\arraystretch{1.5}
    \centering
    \begin{tabular}{|c|c|c|c|c|}
        \hline
        $A_{\text{virt}}$ & Virtuelle Seite              & $A_{\text{phys}}$ &  Offset (virt.) & $A_{\text{real}}$\\
        \hline
        0     & $\lfloor \frac{0}{4096} \rfloor$ = 0      & 24576 & 0 \% 4096     = 0     & 24576 \\
        \hline
        12288 & $\lfloor \frac{12288}{4096} \rfloor$ = 3  & 8192     & 12288 \% 4096 = 0     & 8192 \\
        \hline
        14500  & $\lfloor \frac{14500}{4096} \rfloor$ = 3 & 8192 & 14500 \% 4096 = 2212   & 10404 \\
        \hline
        22000  & $\lfloor \frac{22000}{4096} \rfloor$ = 5 & -    & 22000 \% 4096 = 1520   & \textit{Plattenblock 2}  \\
        \hline
        9100   & $\lfloor \frac{9100}{4096} \rfloor$ = 2  & 0    & 9100 \% 4096 = 908     & 908\\
        \hline
        26758  & $\lfloor \frac{26758}{4096} \rfloor$ = 6 & 4096 & 26758 \% 4096 = 2182   & 6278 \\
        \hline
    \end{tabular}
    \caption{Zuordnung virtueller Adressen zu Hauptspeicheradressen für Aufgabe 4 a). ``\%`` repräsentiert den MODULO-Operator. }
    \label{tab:adresszuordnung_a}
\end{table}


\section{b)}
Der Inhalt der virtuellen Seite $3$ wird in dem Sekundärspeicher auf Plattenblock 17 ausgelagert.\\
Die virtuelle Seite $5$ bekommt nun den Primärspeicher beginnend ab Adresse 8192 zugeordnet.\\
Damit ändert sich die Tabelle wie in  Tabelle~\ref{tab:adresszuordnung_b} angegeben.

\begin{table}[h!]
    \setlength{\tabcolsep}{0.5em}
    \def\arraystretch{1.5}
    \centering
    \begin{tabular}{|c|c|c|c|c|}
        \hline
        $A_{\text{virt}}$ & Virtuelle Seite              & $A_{\text{phys}}$ &  Offset (virt.) & $A_{\text{real}}$\\
        \hline
        0     & $\lfloor \frac{0}{4096} \rfloor$ = 0      & 24576 & 0 \% 4096     = 0     & 24576 \\
        \hline
        12288 & $\lfloor \frac{12288}{4096} \rfloor$ = 3  & -     & 12288 \% 4096 = 0     & \textit{Plattenblock 17} \\
        \hline
        14500  & $\lfloor \frac{14500}{4096} \rfloor$ = 3 & - & 14500 \% 4096 = 2212   & \textit{Plattenblock 17} \\
        \hline
        22000  & $\lfloor \frac{22000}{4096} \rfloor$ = 5 & 8192    & 22000 \% 4096 = 1520   & 9712  \\
        \hline
        9100   & $\lfloor \frac{9100}{4096} \rfloor$ = 2  & 0    & 9100 \% 4096 = 908     & 908\\
        \hline
        26758  & $\lfloor \frac{26758}{4096} \rfloor$ = 6 & 4096 & 26758 \% 4096 = 2182   & 6278 \\
        \hline
    \end{tabular}
    \caption{Zuordnung virtueller Adressen zu Hauptspeicheradressen für Aufgabe 4 b). }
    \label{tab:adresszuordnung_b}
\end{table}

\section{c)}

Die Seitentabelle ist nicht zulässig, da die Seitengröße 4KiB beträgt - also muss jeder Seitenbereich, der durch reale Adressen in der Seitentabelle repräsentiert wird, mit einem vielfachen von 4096 Bytes beginnen.
Der abgebildete Speicherbereich mit 10K-14K erfüllt diese Bedingung nicht.
Zudem überschneidet sich der Eintrag mit dem Speicherbereich 8K-12K - hier würden zwei verschiedene virtuelle Seiten auf ein und dieselbe reale physische Seite verweisen (oder einen Bereich davon).
Handelt es sich bei der geg. Seitentabelle um die eines einzelnen Prozesses, hätte dies das unerwünschte Überschreiben von Adressinhalten zur Folge.\\
Anzumerken ist hierbei jedoch, das durchaus mehrere virtuelle Seiten verschiedener Prozesse auf ein und dieselbe physische Seite abgebildet werden dürfen, bspw. zur Umsetzung von \textit{Shared Memory} (vgl.~\cite[\textbf{Abb. 7.38}, 226]{Man20h}).