\chapter{Aufgabe 5}

\noindent
Der Lösungsvorschlag zu der Aufgabe ist in den Tabellen~\ref{tab:zugriff_fifo} sowie~\ref{tab:zugriff_lru} dargestellt.
Positionen bzw. Positionswechsel der Seiten sind nicht repräsentativ für das Verhalten des Speichers, ausgenommen der Darstellung der Seitenersetzung.

\noindent
Für die Anzahl der \textit{Page Faults} und \textit{Seitenersetzungen} ergibt sich jeweils:

\begin{itemize}
    \itemsep0.5em
    \item \textbf{FIFO}
    \item[] Page Faults: 8 ; Seitenersetzungen: 4
    \item \textbf{LRU}:
    \item[] Page Faults: 6 ; Seitenersetzungen: 2
\end{itemize}

\noindent
Zu beachten ist die Definition eines \textit{Page Faults}:
\blockquote[{\cite[198 f., Hervorhebung i.O.]{Man20h}}]{
``Ein \textit{page fault}  (Seitenfehler) ist also ein Trap [\ldots], den die MMU [Anm.: Memory Management Unit] erzeugt, wenn von einem Prozess eine virtuelle Adresse angesprochen wird, die nicht im Hauptspeicher geladen ist.``
}\\

\noindent
Demnach ist bereits der erste Zugriff auf die geg. virt. Seiten als \textit{Page Fault} zu werten, da der Hauptspeicher zu Beginn \textit{leer} ist.

\begin{table}[h!]
    \setlength{\tabcolsep}{0.6em}
    \def\arraystretch{1.5}
    \centering
    \begin{tabular}{|c|*{11}{c|}}
        \hline
        \textbf{Zugriff} & \textbf{1} & \textbf{2} & \textbf{2} & \textbf{3} & \textbf{4} & \textbf{2} & \textbf{1} & \textbf{5} & \textbf{6} & \textbf{2} & \textbf{1} \\
        \hline
        \multirow{\textbf{Hauptspeicher}}
        &  1_E & 1    & 1 & 1   & 1   & 1 & 1 & 2     & 3     & 4      & 5   \\
        \cline{2-12}
        &      & 2_E  & 2 & 2   & 2   & 2 & 2 & 3     & 4     & 5      & 6   \\
        \cline{2-12}
        &      &      &   & 3_E & 3   & 3 & 3 & 4     & 5     & 6      & 2 \\
        \cline{2-12}
        &      &      &   &     & 4_E & 4 & 4 & 5_E^R & 6_E^R & 2_E^R  & 1^R_E \\
        \hline
    \end{tabular}
    \caption{Ersetzungsverfahren für Aufgabe 5 nach \textbf{FIFO}: Es stehen 4 Seitenrahmen im Hauptspeicher zur Verfügung.
    Ein hochgestelltes $R$ bezeichnet eine Seitenersetzung, ein tiefgestelltes $E$ einen \textit{Page Fault}.}
    \label{tab:zugriff_fifo}
\end{table}

\begin{table}[h!]
    \setlength{\tabcolsep}{0.6em}
    \def\arraystretch{1.5}
    \centering
    \begin{tabular}{|c|*{11}{c|}}
        \hline
        \textbf{Zugriff} & \textbf{1} & \textbf{2} & \textbf{2} & \textbf{3} & \textbf{4} & \textbf{2} & \textbf{1} & \textbf{5} & \textbf{6} & \textbf{2} & \textbf{1} \\
        \hline
        \multirow{\textbf{Hauptspeicher}}
        &  1_E & 1    & 1 & 1   & 1   & 1 & 1 & 1     & 1     & 1      & 1   \\
        \cline{2-12}
        &      & 2_E  & 2 & 2   & 2   & 2 & 2 & 2     & 2     & 2      & 2   \\
        \cline{2-12}
        &      &      &   & 3_E & 3   & 3 & 3 & 5_E^R & 5     & 5      & 5 \\
        \cline{2-12}
        &      &      &   &     & 4_E & 4 & 4 & 4     & 6_E^R & 6      & 6 \\
        \hline
    \end{tabular}
    \caption{Ersetzungsverfahren für Aufgabe 5 nach \textbf{LRU}. Notation wie in Tabelle~\ref{tab:zugriff_fifo}.}
    \label{tab:zugriff_lru}
\end{table}

\clearpage
\section{d)}

Sei folgender Seitenzugriff gegeben:

\begin{equation}\notag
    1 \ 2 \ 3 \ 1 \ 2\ 3 \ 2 \ 2 \ 3
\end{equation}

\noindent
Sei $t^d$ der Zeitpunkt des letzten Zugriffs. Dann ergeben sich für die Distanzen $d \in \{2, 4, 6\}$ jeweils die folgenden \textit{Working Sets} (\textit{WS})$\footnote{
\textit{Working-Set-Modell}: s.~\cite[214 ff.]{Man20h}
}:

\begin{equation}\notag
    \begin{split}
        \text{WS}(t^2) &= \{2, 3\} \\
        \text{WS}(t^4) &= \{2, 3\} \\
        \text{WS}(t^6) &= \{1, 2, 3\} \\
    \end{split}
\end{equation}