%%%%%%%%%%%%%%%%%%% vorlage.tex %%%%%%%%%%%%%%%%%%%%%%%%%%%%%
%
% LaTeX-Vorlage zur Erstellung von Projekt-Dokumentationen
% im Fachbereich Informatik der Hochschule Trier
%
% Basis: Vorlage svmono des Springer Verlags
%
%%%%%%%%%%%%%%%%%%%%%%%%%%%%%%%%%%%%%%%%%%%%%%%%%%%%%%%%%%%%%

\documentclass[envcountsame,envcountchap, deutsch]{i-studis}

\usepackage{footnote}
\usepackage[LGR,T1]{fontenc}
\usepackage{float}
\usepackage{python}
\usepackage{xparse}
\usepackage{enumitem}
\usepackage[outputdir=../../../auxil]{minted}
\usepackage{comment}
\usepackage[normalem]{ulem}
\usepackage[inkscapeformat=png]{svg}
\usepackage[titles]{tocloft}
\usepackage{titlesec}
\usepackage[pdftex,plainpages=false]{hyperref}
\usepackage{xurl}
\usepackage[toc]{glossaries}
\usepackage[titletoc,title,page, header]{appendix} % Anhang
\usepackage[backend=biber, style=alphabetic, isbn=true, doi=true]{biblatex}
\usepackage{makeidx}         	% Index
\usepackage{multicol}% Zweispaltiger Index
\usepackage[threshold=0]{csquotes}
\usepackage{soul}
\usepackage{caption}
\usepackage{makecell}
\usepackage{stix}
\usepackage{tabularx}

\setcounter{tocdepth}{4}
\setcounter{secnumdepth}{6}

\captionsetup{font=small}
\captionsetup{labelfont=bf}

\newcommand{\textgreek}[1]{\begingroup\fontencoding{LGR}\selectfont#1\endgroup}

\BeforeBeginEnvironment{tcolorbox}{\savenotes}
\AfterEndEnvironment{tcolorbox}{\spewnotes}

%\usepackage[bottom]{footmisc}	% Erzeugung von Fu�noten

%%-----------------------------------------------------
%\newif\ifpdf
%\ifx\pdfoutput\undefined
%\pdffalse
%\else
%\pdfoutput=1
%\pdftrue
%\fi
%%--------------------------------------------------------
%\ifpdf
\usepackage[pdftex]{graphicx}
\usepackage{epstopdf}

\usepackage{tcolorbox}
\usepackage[table]{colortbl}% http://ctan.org/pkg/xcolor
%\else
%\usepackage{graphicx}
%\usepackage[plainpages=false]{hyperref}
%\fi

%%-----------------------------------------------------
\usepackage{color}				% Farbverwaltung
%\usepackage{ngerman} 			% Neue deutsche Rechtsschreibung
\usepackage[english, ngerman]{babel}

\renewcommand\appendixpagename{Anhänge}
\renewcommand\appendixtocname{Anhänge}
%%-----------------------------------------------------
% Unterscheidung für Umlaute Windows-Mac
%%-----------------------------------------------------

%\usepackage[latin1]{inputenc} 	% Ermöglicht Umlaute-Darstellung
\usepackage[utf8]{inputenc}  	% Ermöglicht Umlaute-Darstellung unter Linux (je nach verwendetem Format)

%-----------------------------------------------------
\usepackage{listings} 			% Code-Darstellung
\lstset
{
	basicstyle=\ttfamily,
% print whole listing small
	mathescape,
	columns=fullflexible,
	literate=%
		{Ö}{{\"O}}1
		{Ä}{{\"A}}1
		{Ü}{{\"U}}1
		{ß}{{\ss}}1
		{ü}{{\"u}}1
		{ä}{{\"a}}1
		{ö}{{\"o}}1
}

\NewDocumentCommand{\code}{v}{%
	\texttt{\textcolor{blue}{#1}}%
}
\NewDocumentCommand{\staticcode}{v}{%
	\ul{{\texttt{\textcolor{blue}{#1}}}}%
}



\usepackage{textcomp} 			% Celsius-Darstellung
\usepackage{amssymb,amsfonts,amstext,amsmath}	% Mathematische Symbole
\usepackage[german, ruled, vlined]{algorithm2e}
\usepackage[a4paper]{geometry} % Andere Formatierung
\usepackage{bibgerm}
\usepackage{array}
\usepackage{mathtools}
\usepackage{mdwtab}
\usepackage{datagidx}
\usepackage{lstmisc}
\usepackage{wasysym}
\usepackage{booktabs}
\hyphenation{Ele-men-tar-ob-jek-te  ab-ge-tas-tet Aus-wer-tung House-holder-Matrix Le-ast-Squa-res-Al-go-ri-th-men} 		% Weitere Silbentrennung bei Bedarf angeben
\setlength{\textheight}{1.1\textheight}
\pagestyle{myheadings} 			% Erzeugt selbstdefinierte Kopfzeile
\makeindex 						% Index-Erstellung

% uncomment to hide tables, equations and figures
%\excludecomment{confidential}
%\excludecomment{figure}
%\excludecomment{equation}
%\excludecomment{table}
%\let\endfigure\relax
%\let\endtable\relax
%\let\endequation\relax


\DeclareFieldFormat{doi}{\textsc{doi}: \texttt{#1}}
\DeclareFieldFormat{isbn}{\textsc{isbn}: \texttt{#1}}

\DeclareFieldFormat{postnote}{#1}
\DeclareFieldFormat{multipostnote}{#1}

\addbibresource{literatur.bib}


\newglossarystyle{mystyle}{%
	\setglossarystyle{treegroup}%

	\renewcommand*{\glossentry}[2]{%
		\glsentryitem{##1}\textbf{\glstarget{##1}{\glossentryname{##1}}}%
		\par
		\glossentrydesc{##1}
		\par
		\vspace{1cm}
	}%
}

\setglossarystyle{mystyle}



%\makenoidxglossaries
%\loadglsentries{chapters/Glossar/index.tex}

%--------------------------------------------------------------------------
\begin{document}

%------------------------- Titelblatt -------------------------------------
	\title{\begin{center}
			   Einsendeaufgaben ES (3)

			   \textbf{}\\
			   \\

			   \\
			   \small{Modul es, SS25} \\ \small{Trier University of Applied Sciences}\\ \small{Informatik Fernstudium (M.C.Sc.)}
	\end{center}}
	\project{}
%--------------------------------------------------------------------------
	\supervisor{Titel Vorname Name} 		% Betreuer der Arbeit
	\author{\begin{center}

	\end{center}}							% Autor der Arbeit
	\address{\begin{center}
				 \small{03.04.2025\\  Thorsten Suckow-Homberg (\url{thorsten@suckow-homberg.de})}
	\end{center}} 							% Im Zusammenhang mit dem Datum wird hinter dem Ort ein Komma angegeben
	\submitdate{} 				% Abgabedatum
%\begingroup
%  \renewcommand{\thepage}{title}
%  \mytitlepage
%  \newpage
%\endgroup



	\begingroup
	\renewcommand{\thepage}{Titel}
	\mytitlepage
	\newpage
	\endgroup
%--------------------------------------------------------------------------
	\frontmatter
%--------------------------------------------------------------------------

	\tableofcontents 						% Inhaltsverzeichnis
%--------------------------------------------------------------------------
	\mainmatter                        		% Hauptteil (ab hier arab. Seitenzahlen)
%--------------------------------------------------------------------------
% Die Kapitel werden in separaten .tex-Dateien abgelegt und hier eingebunden.
	\chapter{Aufgabe 1}

\section{Teil 1}

\textit{Warum ist es wichtig zu wissen, an welcher Speicherstelle das Programm startet?}\\

\noindent
Wir dürfen für den allgemeinen Fall feststellen: Würde die Information fehlen, an welcher Stelle ein Programm startet,  würde u.a. Speicher nicht richtig initialisiert oder zurückgesetzt. So zeigen \textit{Bollenbacher und Liell} in~\cite[\textbf{Tabelle 19}, 86]{BL22} ein Programm zur Addition zweier Zahlen auf dem M6800\footnote{
\url{https://en.wikipedia.org/wiki/Motorola_6800}, abgerufen 22.03.2025
}, das mit der Instruktion \texttt{CLR A}\footnote{
s. a. \url{http://www.8bit-era.cz/6800.html#CLR}, abgerufen 22.03.2025
} beginnt, und damit den Inhalt des  Akkumulators\footnote{
Register zur Aufnahme von Operanden bzw. Ergebnissen einer Rechenoperation, vgl.~\cite[233]{Fri21})
} löscht.
Würde in diesem Beispiel die Startadresse des Programms nicht bekannt sein, würde - sofern überhaupt eine (sinnvolle) Rechenoperation ausgeführt wird - unter hoher Wahrscheinlichkeit mit Artefakten vorhergehender Rechenoperationen gearbeitet bzw. das Programm in einem fremden Adressraum operieren.\\

\noindent
Insgesamt darf man wohl behaupten, dass ohne Kenntnis der Startadresse eines Programm ein (deterministischer) \textbf{Rechenbetrieb nicht möglich} wäre, zumal darauffolgende anzusteuernde Operationen über den Befehlszähler auf den nächsten auszuführenden Befehl zeigen\footnote{
Ein \textbf{Befehlszähler} als Teil eines \textbf{Leitwerks}\footnote{
    hier insb. bei der Von-Neumann-Rechnerarchitektur
} ist ein Register, der die Adressen des als nächstes auszuführenden Maschinen-Befehls enthält (vgl.~\cite[234]{Fri21}).
}, und sich daraus ja gerade die auszuführenden Rechneoperationen ergeben.\\

\noindent
Abstrahieren wir ausgehend von dem Fall des Assemblerprogramms weiter, können wir die Startadresse wie den \textit{Startzustand} eines Turingautomaten verstehen, der entsprechend einer geg. Zustandsüberführungsfunktion $\delta$ (dem \textit{Turingprogramm}, vgl.~\cite[269 f.]{VW16h}) eine Eingabe verarbeitet.

\section{Teil 2}

Der Lösungsvorschlag ist in Tabelle~\ref{tab:speicherinstruktionen} angegeben.

\begin{table}[h!]
    \setlength{\tabcolsep}{0.5em}
    \def\arraystretch{1.5}
    \centering
    \begin{tabular}{|c|c|l|l|}
        \hline
        \textbf{Zeile} & \textbf{Speicheradresse} & \multicolumn{2}{c|}{\textbf{Instruktion}} \\
        \cline{3-4}
        &                          & \textbf{Hex-Code} & \textbf{Mnemonics (Kürzel)} \\
        \hline
        $1$ & $0000$ & \texttt{4f} & \texttt{start CLR A}  \\

        $2$ & $0001$ & \texttt{b6 00 14} & \qquad\quad\texttt{LDA A value}  \\

        $3$ & $0004$ & \texttt{b7 00 12} & \qquad\quad\texttt{STA A result1}  \\

        $4$ & $0007$ & \texttt{8b 0f} & \qquad\quad\texttt{ADD A \#15}  \\

        $5$ & $0009$ & \texttt{7e 00 0c} & \qquad\quad\texttt{JMP next}  \\

        $6$ & $000c$ & \texttt{b7 00 13} & \texttt{next STA A result2}  \\

        $7$ & $000f$ & \texttt{7e 00 00} & \qquad\quad\texttt{JMP start}\\
        &&&\\
        $8$ & $0012$ & \texttt{00} & \texttt{result1 byte 00}  \\
        $9$ & $0013$ & \texttt{00} & \texttt{result2 byte 00}  \\
        $10$ & $0014$ & \texttt{40} & \texttt{value byte 40}  \\

        \hline
    \end{tabular}
    \caption{Lösungsvorschlag zu Aufgabe 1.2}
    \label{tab:speicherinstruktionen}
\end{table}


\section{Teil 3}

\textit{(1) Welche Speicherstellen werden für die Ablage von Instruktionen
verwendet?}\\

\noindent
Es werden die folgenden Speicherstellen für die Ablage von Instruktionen verwendet:

\begin{itemize}
    \itemsep0.5em
    \item \texttt{0000}
    \item \texttt{0001}
    \item \texttt{0004}
    \item \texttt{0007}
    \item \texttt{0009}
    \item \texttt{000c}
    \item \texttt{000f}
\end{itemize}

\vspace{5mm}

\noindent
\textit{(2) Welche Speicherstellen werden für die Ablage von Daten verwendet?}\\

\noindent
Es werden die folgenden Speicherstellen für die Ablage von Daten verwendet:

\begin{itemize}
    \itemsep0.5em
    \item \texttt{0012}
    \item \texttt{0013}
    \item \texttt{0014}
\end{itemize}

\section{Teil 4}

\noindent
\textit{Welche Rechnung führt das Programm aus?}\\

\noindent
Überführung in C-Code zur \underline{Veranschaulichung}:

\begin{minted}[mathescape,
    numbersep=5pt,
    gobble=2,
    frame=none,
    framesep=2mm]{c}

    int A       = 0x00; // Hilfsvariable für Akkumulator
    int result1 = 0x00; // result1 byte 00
    int result2 = 0x00; // result2 byte 00
    int value   = 0x40;   // value   byte 40

    start:
        A        = 0x00;  // CLR A
        A        = value; // LDA A value
        result1  = A;     // STA A result1
        A       += 15;    // ADD A#15
        goto next;        // JMP next

    next:
        result2 = A;      // STA A result 2
        goto start;       // JMP start
\end{minted}\\

\noindent
\begin{itemize}
    \itemsep0.5em
    \item \texttt{start}
    \item[] In dem Programmteil werden die Variablen initialisiert.
    In den Akkumulator \code{A} wird der Wert von \code{value} geschrieben; da sich \code{value} während der Programmausführung nie ändert, wird \code{A} im weiteren Programmablauf an dieser Stelle - da durch \code{CLR A}  der Inhalt des Akkumulators vorher gelöscht wird - immer mit dem Wert $64$ (dezimal) belegt.\\
    Der Inhalt des Akkumulators wird daraufhin in die durch \code{result1} repräsentierte Speicherstelle geschrieben.\\
     Anschliessen wird zu dem Wert in \code{A} der Wert $15$ (dezimal) hinzuaddiert, das Ergebnis entspricht etwa \code{A = 79}.\\
    \code{result1} bleibt indes unverändert.
    \item \texttt{next}
    \item[] Der Programmteil \texttt{start} springt in den durch den Identifier ausgezeichneten Programmteil \texttt{next}.\\
    Hier wird der Wert von \code{A} in die Speicherstelle von \code{result2} geschrieben (entspricht etwa \code{result2 = 79}).\\
    Danach wird zurück in den Programmteil \texttt{start} gesprungen, die Ausführung wiederholt sich.
\end{itemize}


	%\glsaddall
	%\printnoidxglossary[title=Glossar, toctitle=Glossar]

	%\chapter{Aufgabe 1}

\section{Teil 1}

\textit{Warum ist es wichtig zu wissen, an welcher Speicherstelle das Programm startet?}\\

\noindent
Wir dürfen für den allgemeinen Fall feststellen: Würde die Information fehlen, an welcher Stelle ein Programm startet,  würde u.a. Speicher nicht richtig initialisiert oder zurückgesetzt. So zeigen \textit{Bollenbacher und Liell} in~\cite[\textbf{Tabelle 19}, 86]{BL22} ein Programm zur Addition zweier Zahlen auf dem M6800\footnote{
\url{https://en.wikipedia.org/wiki/Motorola_6800}, abgerufen 22.03.2025
}, das mit der Instruktion \texttt{CLR A}\footnote{
s. a. \url{http://www.8bit-era.cz/6800.html#CLR}, abgerufen 22.03.2025
} beginnt, und damit den Inhalt des  Akkumulators\footnote{
Register zur Aufnahme von Operanden bzw. Ergebnissen einer Rechenoperation, vgl.~\cite[233]{Fri21})
} löscht.
Würde in diesem Beispiel die Startadresse des Programms nicht bekannt sein, würde - sofern überhaupt eine (sinnvolle) Rechenoperation ausgeführt wird - unter hoher Wahrscheinlichkeit mit Artefakten vorhergehender Rechenoperationen gearbeitet bzw. das Programm in einem fremden Adressraum operieren.\\

\noindent
Insgesamt darf man wohl behaupten, dass ohne Kenntnis der Startadresse eines Programm ein (deterministischer) \textbf{Rechenbetrieb nicht möglich} wäre, zumal darauffolgende anzusteuernde Operationen über den Befehlszähler auf den nächsten auszuführenden Befehl zeigen\footnote{
Ein \textbf{Befehlszähler} als Teil eines \textbf{Leitwerks}\footnote{
    hier insb. bei der Von-Neumann-Rechnerarchitektur
} ist ein Register, der die Adressen des als nächstes auszuführenden Maschinen-Befehls enthält (vgl.~\cite[234]{Fri21}).
}, und sich daraus ja gerade die auszuführenden Rechneoperationen ergeben.\\

\noindent
Abstrahieren wir ausgehend von dem Fall des Assemblerprogramms weiter, können wir die Startadresse wie den \textit{Startzustand} eines Turingautomaten verstehen, der entsprechend einer geg. Zustandsüberführungsfunktion $\delta$ (dem \textit{Turingprogramm}, vgl.~\cite[269 f.]{VW16h}) eine Eingabe verarbeitet.

\section{Teil 2}

Der Lösungsvorschlag ist in Tabelle~\ref{tab:speicherinstruktionen} angegeben.

\begin{table}[h!]
    \setlength{\tabcolsep}{0.5em}
    \def\arraystretch{1.5}
    \centering
    \begin{tabular}{|c|c|l|l|}
        \hline
        \textbf{Zeile} & \textbf{Speicheradresse} & \multicolumn{2}{c|}{\textbf{Instruktion}} \\
        \cline{3-4}
        &                          & \textbf{Hex-Code} & \textbf{Mnemonics (Kürzel)} \\
        \hline
        $1$ & $0000$ & \texttt{4f} & \texttt{start CLR A}  \\

        $2$ & $0001$ & \texttt{b6 00 14} & \qquad\quad\texttt{LDA A value}  \\

        $3$ & $0004$ & \texttt{b7 00 12} & \qquad\quad\texttt{STA A result1}  \\

        $4$ & $0007$ & \texttt{8b 0f} & \qquad\quad\texttt{ADD A \#15}  \\

        $5$ & $0009$ & \texttt{7e 00 0c} & \qquad\quad\texttt{JMP next}  \\

        $6$ & $000c$ & \texttt{b7 00 13} & \texttt{next STA A result2}  \\

        $7$ & $000f$ & \texttt{7e 00 00} & \qquad\quad\texttt{JMP start}\\
        &&&\\
        $8$ & $0012$ & \texttt{00} & \texttt{result1 byte 00}  \\
        $9$ & $0013$ & \texttt{00} & \texttt{result2 byte 00}  \\
        $10$ & $0014$ & \texttt{40} & \texttt{value byte 40}  \\

        \hline
    \end{tabular}
    \caption{Lösungsvorschlag zu Aufgabe 1.2}
    \label{tab:speicherinstruktionen}
\end{table}


\section{Teil 3}

\textit{(1) Welche Speicherstellen werden für die Ablage von Instruktionen
verwendet?}\\

\noindent
Es werden die folgenden Speicherstellen für die Ablage von Instruktionen verwendet:

\begin{itemize}
    \itemsep0.5em
    \item \texttt{0000}
    \item \texttt{0001}
    \item \texttt{0004}
    \item \texttt{0007}
    \item \texttt{0009}
    \item \texttt{000c}
    \item \texttt{000f}
\end{itemize}

\vspace{5mm}

\noindent
\textit{(2) Welche Speicherstellen werden für die Ablage von Daten verwendet?}\\

\noindent
Es werden die folgenden Speicherstellen für die Ablage von Daten verwendet:

\begin{itemize}
    \itemsep0.5em
    \item \texttt{0012}
    \item \texttt{0013}
    \item \texttt{0014}
\end{itemize}

\section{Teil 4}

\noindent
\textit{Welche Rechnung führt das Programm aus?}\\

\noindent
Überführung in C-Code zur \underline{Veranschaulichung}:

\begin{minted}[mathescape,
    numbersep=5pt,
    gobble=2,
    frame=none,
    framesep=2mm]{c}

    int A       = 0x00; // Hilfsvariable für Akkumulator
    int result1 = 0x00; // result1 byte 00
    int result2 = 0x00; // result2 byte 00
    int value   = 0x40;   // value   byte 40

    start:
        A        = 0x00;  // CLR A
        A        = value; // LDA A value
        result1  = A;     // STA A result1
        A       += 15;    // ADD A#15
        goto next;        // JMP next

    next:
        result2 = A;      // STA A result 2
        goto start;       // JMP start
\end{minted}\\

\noindent
\begin{itemize}
    \itemsep0.5em
    \item \texttt{start}
    \item[] In dem Programmteil werden die Variablen initialisiert.
    In den Akkumulator \code{A} wird der Wert von \code{value} geschrieben; da sich \code{value} während der Programmausführung nie ändert, wird \code{A} im weiteren Programmablauf an dieser Stelle - da durch \code{CLR A}  der Inhalt des Akkumulators vorher gelöscht wird - immer mit dem Wert $64$ (dezimal) belegt.\\
    Der Inhalt des Akkumulators wird daraufhin in die durch \code{result1} repräsentierte Speicherstelle geschrieben.\\
     Anschliessen wird zu dem Wert in \code{A} der Wert $15$ (dezimal) hinzuaddiert, das Ergebnis entspricht etwa \code{A = 79}.\\
    \code{result1} bleibt indes unverändert.
    \item \texttt{next}
    \item[] Der Programmteil \texttt{start} springt in den durch den Identifier ausgezeichneten Programmteil \texttt{next}.\\
    Hier wird der Wert von \code{A} in die Speicherstelle von \code{result2} geschrieben (entspricht etwa \code{result2 = 79}).\\
    Danach wird zurück in den Programmteil \texttt{start} gesprungen, die Ausführung wiederholt sich.
\end{itemize}

%\input{chapters/ZusammenfassungAusblick}
% ...
%--------------------------------------------------------------------------
	\backmatter                        		% Anhang
%-------------------------------------------------------------------------

%\bibliographystyle{geralpha}			% Literaturverzeichnis
%\bibliography{literatur}     			% BibTeX-File literatur.bib
%\raggedright
	\sloppy
	\printbibliography
	\fussy

	\printindex


\end{document}
