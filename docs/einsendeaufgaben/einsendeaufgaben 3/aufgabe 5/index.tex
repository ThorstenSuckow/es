\chapter{Aufgabe 5}


\section{Teil 1}

\textit{Was ist ein ``Port``? Was ist der Unterschied zu einem ``Register``? Wie
viele Ports hat die Parallelschnittstelle?}\\

\noindent
Ein \textbf{Port} bezeichnet einen physikalischen Speicherbereich, der dazu dient, Daten von Treibern entgegenzunehmen bzw. von Treibern auszulesen, wobei die Speicherbereiche von dem Treiber beschrieben werden und von angeschlossenen Geräten gelesen werden, und umgekehrt:
Ein Port lässt sich grob als ``Schleuse`` / ``Schnittstelle`` für den Datenfluss zwischen Software und Hardware betrachten.\\

\noindent
Ein \textbf{Register} in der Soft-/Hardware-Entwicklung ist im Allgemeinen als eine \textit{Speicherzelle} zum Ablegen und Auslesen von Daten zu verstehen.

\blockquote[{\cite[54]{ES3}}]{
    ``Sind die Register an einen speziellen (IO-)Bus angeschlossen, werden sie auch als Port bezeichnet.``
}\\

\noindent
Ein Port kann mehrere dieser Register zusammenfassen, wie \textit{Tanenbaum und Bos} schreiben:

\blockquote[{\cite[31, Hervorhebung i.O.]{TB24}}]{
[\ldots] a minimal disk controller might have registers for specifying
    the disk address, memory address, sector count, and direction (read or write) [\ldots]
    The collection of all the device registers forms the \textbf{I/O port space} [\ldots]
}

\noindent
Eine \textbf{Parallelschnittstelle} nach \textit{IEEE 1284}\footnote{
\url{https://en.wikipedia.org/wiki/IEEE_1284}, abgerufen 06.04.2025
} verfügt über 3 Ports: Datenport, Statusport, Controlport (vgl.~\cite[54 f.]{ES3}):

\blockquote[{IEEE 1284: Parallel Ports\footnote{\url{https://web.archive.org/web/20061115015448/http://www.nor-tech.com/solutions/dox/ieee1284_parallel_ports.pdf}, abgerufen 06.04.2025}}]{
``The standard parallel port uses three consecutive addresses. The
first is the base address or Data register, the second is the port’s Status register, and the
    third is the port’s Control register.``
}


\section{Teil 2}

Der Quelltext des Programms ist im Folgenden angegeben.\\
Die Lösung verwendet zur Darstellung des 256 Bit-Feldes ein \code{unsigned char}-Feld der Länge \code{BITMAP_SIZE}\footnote{
die Makrodefinition von \texttt{BYTESIZE} dient hier alleine zur Vermeidung von Tippfehlern u.ä. im Quellcode.
}.\\

\noindent
Für die Darstellung der Bits werden die im Feld gespeicherten Bytes (\code{0x00}) beim (Zurück-)Setzen jeweils UND- bzw. ODER-maskiert, wobei zum Zurücksetzen eine invertierte Bitmaske verwendet wird - Details finden sich in der Funktion \code{_BitMapSet}, die die Logik von \code{BitMapSet} und \code{BitMapClear} zusammenfasst.

\begin{minted}[mathescape,
    numbersep=5pt,
    gobble=2,
    frame=none,
    framesep=2mm,
    fontsize=\small]{c}
    #include <stdio.h>
    #include <stdlib.h>

    #define BITMAP_SIZE 32
    #define BYTESIZE 8


    unsigned char bitmap[BITMAP_SIZE];


    void printBitmap(int bit_i, int set) {

        if (set == 0) {
            printf("Clearing bit %d:", bit_i);
        } else {
            printf("Setting bit %d:", bit_i);
        }

        for (int i = BITMAP_SIZE - 1; i >= 0; i--) {
            for (int j = 0; j < BYTESIZE; j++) {
                if (j % (BYTESIZE / 2) == 0) {
                    printf(" ");
                }
                printf("%d", (bitmap[i] & (0x01 << (BYTESIZE - 1) - j)) ? 1 : 0);

            }
        }
        printf("\n");
    }


    void BitMapInit() {
        for (int i = 0; i < BITMAP_SIZE; i++) {
            bitmap[i] = 0x00;
        }
    }


    void _BitMapSet(int PortNbr, int value) {
        if (value != 0 && value != 1) {
            fprintf(stderr, "Unexpected value for bit: %d\n", value);
            exit(1);

        }

        int idx = PortNbr / BYTESIZE;

        if (PortNbr < 0 || idx > BITMAP_SIZE -1) {
            fprintf(stderr,
            "Error: PortNbr %d does not exist in range [0, %d]\n",
            PortNbr, BITMAP_SIZE * BYTESIZE - 1);
            exit(1);
        }

        int newbit = 0x01 << (PortNbr % 8);

        if (value == 0) {
            // 0001 0000 -> 1110 1111
            // 1110 1111 & 0101 0001 -> 0100 0001
            bitmap[idx] &= ~newbit;
        } else {
            // 0001 0000 | 0100 0001 -> 0101 0001
            bitmap[idx] |= newbit;
        }
    }


    void BitMapSet(int PortNbr) {
        _BitMapSet(PortNbr, 1);
    }

    void BitMapClear(int PortNbr) {
        _BitMapSet(PortNbr, 0);
    }


    int main(int argc, char *argv[]) {
        if (argc <= 1) {
            printf("Usage: BitMap bit_1 bit_2 ... bit_n, with bit_i in [0,  %d]\n",
                BITMAP_SIZE * BYTESIZE);
            exit(0);
        }

        BitMapInit();
        int bit_i;

        for (int i = 1; i < argc; i++) {
            if (sscanf(argv[i], "%d", &bit_i) != 1) {
                printf("no valid argument for bit_%d, skipping...\n", i - 1);
                continue;
            }

            BitMapSet(bit_i);
            printBitmap(bit_i, 1);
        }

        for (int i = 1; i < argc; i++) {
            if (sscanf(argv[i], "%d", &bit_i) != 1) {
                continue;
            }

            BitMapClear(bit_i);
            printBitmap(bit_i, 0);
        }

        return 0;
    }

\end{minted}

\noindent
Für eine \code{BITMAP_SIZE} von $2$ (= $16$ bit) liefert ein Aufruf des Programms mit den Argumenten
\begin{center}
    \texttt{./BitMap.out 1 6 7 11 15}
\end{center}

\noindent
folgende Ausgabe:

\begin{verbatim}
Setting bit 1: 0000 0000 0000 0010
Setting bit 6: 0000 0000 0100 0010
Setting bit 7: 0000 0000 1100 0010
Setting bit 11: 0000 1000 1100 0010
Setting bit 15: 1000 1000 1100 0010
Clearing bit 1: 1000 1000 1100 0000
Clearing bit 6: 1000 1000 1000 0000
Clearing bit 7: 1000 1000 0000 0000
Clearing bit 11: 1000 0000 0000 0000
Clearing bit 15: 0000 0000 0000 0000
\end{verbatim}