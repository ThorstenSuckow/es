\chapter{Aufgabe 1}

\textit{Schreiben Sie einen Treiber \texttt{PrtDrvr}, der lediglich eine \texttt{write}-Funktion implementiert.
Wenn ein Prozess auf diesen Treiber schreibt, dann sollen in der Datei \texttt{/var/log/system.log} folgende Informationen ausgegeben werden:
\begin{enumerate}
    \itemsep0.5em
    \item Der Name, die PID und die auf die Boot-Zeit bezogene Laufzeit des
    aufrufenden Prozesses.
    \item Der Name, die PID und die auf die Boot-Zeit bezogene Laufzeit aller
    Elternprozesse des aufrufenden Prozesses.
    \item Der Name und die auf die Boot-Zeit bezogene Laufzeit des Prozesses
    mit der PID 1.
\end{enumerate}
}

\section{Implementierung}

Nachfolgend ist der C-Code angegeben.\\
Entwickelt und getestet auf der von der Kurseinheit bereitgestellten VM\footnote{
\texttt{uname -a: Linux hbo-VirtualBox 5.4.0-91-generic #102-Ubuntu SMP Fri Nov 5 16:31:28 UTC 2021 x86\_64 x86\_64 x86\_64 GNU/Linux}
}
Bei der Implementierung stützen wir uns überwiegend auf die Ausführungen in~\cite[91 f. und 95 f.]{ES3}

\begin{minted}[mathescape,
    numbersep=5pt,
    gobble=2,
    frame=none,
    framesep=2mm,
    fontsize=\small]{c}
    #include <linux/module.h>
    #include <linux/version.h>
    #include <linux/proc_fs.h>

    #define LICENSE "GPL"
    #define DRIVER_AUTHOR "Thorsten Suckow-Homberg <thorsten@suckow-homberg.de>"
    #define DRIVER_NAME  "PrtDrvr"

    MODULE_LICENSE(LICENSE);
    MODULE_AUTHOR(DRIVER_AUTHOR);

    static unsigned int prtdrvr_major = 0;

    ssize_t prtdrvr_write(struct file *filp, const char *buf, size_t count, loff_t *t) {

        int i = 1;
        struct task_struct *parent = current->real_parent;

        printk(KERN_INFO "%s_write called. Here's the required  information according to ES3.1:\n"
                         "1. [caller]: %s; PID: %d; start-time: %llu\n",
            DRIVER_NAME,
            current->comm, current->pid, current->real_start_time
        );


        while (true) {
            pid_t pid = parent->pid;

            if (pid == 1) {
                printk(
                    KERN_INFO "3.[systemd] found %s; PID: %d; start-time: %llu\n",
                    parent->comm, pid, parent->real_start_time
                );
                printk(KERN_INFO "(omitting IDLE/SWAPPER. Done!)");

                break;
            } else {
                printk(
                    KERN_INFO "    2.%d [parent] %s; PID: %d; start-time: %llu\n",
                    i, parent->comm, pid, parent->real_start_time
                );
            }

            parent = parent->real_parent;
            i++;
        }

        return count;
    }

    struct file_operations prtdrvr_fops = {
        .owner   = THIS_MODULE,
        .write   = prtdrvr_write
    };

    int init_module() {
        int result = register_chrdev(prtdrvr_major, DRIVER_NAME, &prtdrvr_fops);

        if (result < 0) {
            return result;
        }

        // assign major number from kernel
        if (prtdrvr_major == 0) {
            prtdrvr_major = result;
        }

        // newlines help with flushing ob
        printk(KERN_INFO "%s@%d driver loaded\n", DRIVER_NAME, prtdrvr_major);
        return 0;
    }

    void cleanup_module() {
        unregister_chrdev(prtdrvr_major, DRIVER_NAME);
        printk(KERN_INFO "%s@%d driver unloaded. Bybebye!\n", DRIVER_NAME, prtdrvr_major);
        return;
    }
\end{minted}\\


\section{Test}

\noindent
Getestet wurde der Treiber über die Kommandozeile mit

\begin{center}
\texttt{
dd of=/dev/PrtDrvr if=input
}
\end{center}

\noindent
wobei \texttt{input} eine Datei mit wenigen Bytes Inhalt ist:
Der Input wird an die Device-Datei \texttt{/dev/PrtDrvr} übergeben und dort mittels \texttt{write} verarbeitet.


\section{Ausgabe}
Exemplarisch eine Ausgabe aus \textt{var/log/syslog} nach dem Aufruf von

\begin{center}
    \texttt{dd of=/dev/PrtDrvr if=input}\footnote{
der Übersicht halber um einige Informationen aus \texttt{KERN\_INFO} gekürzt.
}
\end{center}
Die Ausgaben wurden zur einfacheren Zuordnung zu den Aufgabenteilen nummeriert.\\


\noindent
\texttt{kernel: [ 1737.483839] PrtDrvr\_write called. Here's the required  information according to ES3.1:\\
kernel: [ 1737.483839] 1. [caller]: dd; PID: 7602; start-time: 1738292206050\\
kernel: [ 1737.483840]     2.1 [parent] bash; PID: 1763; start-time: 49145654871\\
kernel: [ 1737.483841]     2.2 [parent] gnome-terminal-; PID: 1724; start-time: 45966569883\\
kernel: [ 1737.483842]     2.3 [parent] systemd; PID: 1006; start-time: 20789307747\\
kernel: [ 1737.483842] 3.[systemd] found systemd; PID: 1; start-time: 852230997\\
}
