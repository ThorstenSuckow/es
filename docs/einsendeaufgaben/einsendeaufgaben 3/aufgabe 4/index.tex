\chapter{Aufgabe 4}


\section{Teil 1}

\textit{Mit den Funktionen \texttt{time} und \texttt{localtime} können Sie eine Struktur
Mydate füllen, die die aktuelle Zeit und das aktuelle Datum beinhaltet.
Was machen \texttt{time} und \texttt{localtime} genau? Welche Bedeutung haben die
Komponenten von \texttt{Mydate}?}\\

\subsection*{\texttt{time\_t}}
\texttt{time\_t} ist ein arithmetischer Datentyp zur Speicherung der \textbf{Zeit} - hierfür eignet sich ein ganzzahliger Datentyp zur Repräsentation der Zeit in Sekunden\footnote{
s. \url{https://en.cppreference.com/w/cpp/chrono/c/time_t}, abgerufen 05.04.2025
}, die seit einem festgelegten Datum verstrichen sind.
In diesem Zusammenhang spricht man auch von der \textit{Unix Time} \footnote{
\url{https://en.wikipedia.org/wiki/Unix_time}, abgerufen 05.04.2025
}.

\subsection*{\texttt{time}}
Die Funktion \texttt{time\_t time(time\_t *tp)} liefert die aktuelle Kalenderzeit zurück, oder \texttt{-1}, falls die Referenzzeit nicht zur Verfügung steht.\\
Ist \texttt{tp} nicht NULL, wird der Rückgabewert dem übergebenen Parameter zugewiesen (\cite[256]{KR88}).\\

\noindent
Zu ``Kalenderzeit`` sagt der C-Standard\footnote{
    Quelle: ``final draft of C99 standard plus TC1, TC2, TC3); WG14; 2007``, \url{https://www.open-std.org/jtc1/sc22/wg14/www/docs/n1256.pdf} (abgerufen 05.04.2025)
}:

\blockquote[{\textbf{7.23.1 Components of time}, Hervorhebung i.O.}]{
``Many functions deal with a \textit{calendar time} that represents the current
date (according to the Gregorian calendar) and time. Some functions deal with \textit{local
    time}, which is the calendar time expressed for some specific time zone, and with \textit{Daylight
    Saving Time}, which is a temporary change in the algorithm for determining local time.
    The local time zone and Daylight Saving Time are implementation-defined.``
}\\

\noindent
In C basiert die Kalenderzeit also auf dem \textit{Gregorianischen Kalender}\footnote{
\url{https://en.wikipedia.org/wiki/Gregorian_calendar}, abgerufen 05.04.2025
}.

\subsection*{\texttt{struct tm}}
\texttt{struct}-Definition zur Speicherung einzelner \textit{Komponenten} einer Kalenderzeit (vgl.~\cite[255]{KR88}):

\begin{itemize}
    \itemsep0.5em
    \item \texttt{int tm\_sec}; Sekunden nach Minute (0,61)\footnote{
    ``$61$`` zur Repräsentation von \textit{Leap-Seconds}, s. \url{https://en.wikipedia.org/wiki/Leap_second} (abgerufen 05.04.2025)
    }
    \item \texttt{int tm\_min}; Minuten nach der Stunde (0,59)
    \item \texttt{int tm\_hour}; Stunden seit Mitternacht (0,23)
    \item \texttt{int tm\_mday}; Tag des Monats (1,31)
    \item \texttt{int tm\_mon}; Monate seit Januar (0,11)
    \item \texttt{int tm\_year}; Jahre seit 1900
    \item \texttt{int tm\_wday}; Tage seit Sonntag (0,6)
    \item \texttt{int tm\_yday}; Tage seit Januar 1 (0,365)
    \item \texttt{int tm\_isdst}; Daylight Saving Time flag
\end{itemize}

\subsection*{\texttt{localtime}}
Die Funktion \texttt{struct tm localtime(time\_t *tp)} überführt die übergebene Kalenderzeit in die \textit{lokale Zeit} (\cite[256]{KR88}).


\section{Teil 2}

Wir verzichten bei der Lösung auf \texttt{strftime} und geben eine Lösung an, die den übergebenen String nach entsprechenden Vorkommen der Steuerzeichen zeichenweise untersucht.
Die Ausgabe erfolgt ebenfalls zeichenweise.


\begin{minted}[mathescape,
    numbersep=5pt,
    gobble=2,
    frame=none,
    framesep=2mm,
    fontsize=\small]{c}
    #include <stdio.h>
    #include <stdlib.h>
    #include <unistd.h>
    #include <time.h>
    #include <string.h>

    struct tm *MyDate;
    time_t  tim;

    const char *MONTHS[12] = {
        "Januar",
        "Februar",
        "März",
        "April",
        "Mail",
        "Juni",
        "Juli",
        "August",
        "September",
        "Oktober",
        "November",
        "Dezember",
    };

    void pad(int cmp, char *buf) {
        buf[0] = (cmp / 10) + '0';
        buf[1] = (cmp % 10) + '0';
        buf[2] = '\0';
     }

    /**
     * Pads integer values as required (leading 0s) and writes the result in buf.
     */
    int padComponent(const char c, struct tm *MyDate, char *buf) {
        switch (c) {
            case 'S': pad(MyDate->tm_sec, buf); break;
            case 'M': pad(MyDate->tm_min, buf); break;
            case 'H': pad(MyDate->tm_hour, buf);break;
            case 'D': pad(MyDate->tm_mday, buf); break;
            case 'n': pad(MyDate->tm_mon, buf); break;
            case 'y': pad((MyDate->tm_year) % 100, buf);break;
            case 'Y': sprintf(buf, "%d", MyDate->tm_year + 1900);break;
            default:
                return 0;
        }

        return 1;
    }

    /**
     * Returns the number of expected characters in the replacement
     * for the control character c.
     */
    int reqsize(char c) {
        if (c == 'Y') {
            return 4;
        }
        char *seq = "SMHDny";
        char *ptr = strchr(seq, c);
        return ptr != NULL ? 2 : 0;
    }

    int main(int argc, char *argv[]) {

        if (argc == 1) {
            return 0;
        }

        tim = time(NULL);
        MyDate = localtime(&tim);

        // reserve 16 Bytes for the string buffer
        char BUFFER[16];
        int CMD_LENGTH = strlen(argv[1]);

        for (int i = 0; i < CMD_LENGTH; i++) {
            char c = argv[1][i];
            if (c != '%' || i == CMD_LENGTH - 1) {
                printf("%c", c);
                continue;
            }
            char next = argv[1][i + 1];

            // Replacement-logic starts here.
            // skip one ahead
            i++;

            // string-print month-name
            if (next == 'N') {
                printf("%s", MONTHS[MyDate->tm_mon]);
                continue;
            }

            // char not in substitution list;
            // print previous and next char; this will
            // also take care of %%
            if (reqsize(next) == 0) {
                printf("%c%c", c, next);
                continue;
            }

            // pad as required, print, reset BUFFER
            padComponent(next, MyDate, BUFFER);
            printf("%s", BUFFER);
            BUFFER[0] = '\0';
        }
        printf("\n");

        return 0;
    }
\end{minted}\\

\noindent
Die Ausgabe entspricht für die Beispiel-Eingabe \code{Zeit=%H:%M_im_%N_des_Jahres_%Y} aus der Aufgabenstellung:

\begin{center}
    \texttt{Zeit=19:08\_im\_April\_des\_Jahres\_2025}
\end{center}
