\chapter{Aufgabe 3}


\section{Teil 1}
\textit{Schreiben Sie ein C-Programm ListSize.c, das auf dem Bildschirm die
        Größe der Datentypen
\texttt{short},
\texttt{unsigned short},
\texttt{signed short},
\texttt{int},
\texttt{signed int},
\texttt{unsigned int},
\texttt{long},
\texttt{unsigned long},
\texttt{signed long},
\texttt{long long},
\texttt{unsigned long long},
\texttt{signed long long},
\texttt{float},
\texttt{double},
\texttt{long double}
in Byte ausgibt.}\\


\noindent
C-Code zur zeilenweise Ausgabe der Größe (Byte) der einzelnen Datentypen.
Für die \texttt{printf}-Ausgabe verwenden wir für die Substitution des Rückgabewerts von \code{sizeof} (C-Typ \code{size_t}) den Modifier \code{zu}\footnote{\texttt{z} gehört seit ANSI C99 zum Sprachumfang als Modifier für \texttt{signed size\_t}, vgl.~\url{https://en.cppreference.com/w/c/99} (abgerufen 04.04.2025)
}.

\begin{minted}[mathescape,
    numbersep=5pt,
    gobble=2,
    frame=none,
    framesep=2mm,
    fontsize=\small]{c}
    #include <stdio.h>

    void main() {

        printf("%18s (%zu)\n", "short",          sizeof(short));
        printf("%18s (%zu)\n", "unsigned short", sizeof(unsigned short));
        printf("%18s (%zu)\n", "signed short",   sizeof(signed short));

        printf("%18s (%zu)\n", "int",          sizeof(int));
        printf("%18s (%zu)\n", "signed int",   sizeof(signed int));
        printf("%18s (%zu)\n", "unsigned int", sizeof(unsigned int));

        printf("%18s (%zu)\n", "long",          sizeof(long));
        printf("%18s (%zu)\n", "unsigned long", sizeof(unsigned long));
        printf("%18s (%zu)\n", "signed long",   sizeof(signed long));

        printf("%18s (%zu)\n", "long long",          sizeof(long long));
        printf("%18s (%zu)\n", "unsigned long long", sizeof(unsigned long long));
        printf("%18s (%zu)\n", "signed long long",   sizeof(signed long long));

        printf("%18s (%zu)\n", "float",       sizeof(float));
        printf("%18s (%zu)\n", "double",      sizeof(double));
        printf("%18s (%zu)\n", "long double", sizeof(long double));
    }
\end{minted}\\


\noindent
Die Ausgabe entspricht etwa\footnote{
\texttt{Linux hbo-VirtualBox 5.4.0-91-generic #102-Ubuntu SMP Fri Nov 5 16:31:28 UTC 2021 x86\_64 x86\_64 x86\_64 GNU/Linux}
}

\begin{verbatim}
             short (2)
    unsigned short (2)
      signed short (2)
               int (4)
        signed int (4)
      unsigned int (4)
              long (8)
     unsigned long (8)
       signed long (8)
         long long (8)
unsigned long long (8)
  signed long long (8)
             float (4)
            double (8)
       long double (16)
 \end{verbatim}


\section{Teil 2}

Folgendes Makro ist gegeben:

\begin{minted}[mathescape,
    numbersep=5pt,
    gobble=2,
    frame=none,
    framesep=2mm]{c}
    #define min(x,y) ({ \
        typeof(x) _x = (x);	    \
        typeof(y) _y = (y); 	\
        (void) (&_x == &_y);	\
        _x < _y ? _x : _y; })
\end{minted}\\

\noindent
Die Funktionswiese ist wie folgt:\\

\begin{itemize}
    \itemsep0.5em
    \item Der Compiler ersetzt Vorkommen von \texttt{min(x, y)} mit der rechten Seite der Makrodefinition.
    \item Dabei werden temporäre Variablen \code{_x} / \code{_y} eingeführt, die entsprechend \code{x} / \code{y} getypt werden (\code{typeof}).
    \item \code{(void) (&_x == &_y)} führt eine Typkompatibilitätsprüfung durch - hier werden die \textbf{Zeiger-Typen} verglichen.
    Der Compiler gibt eine Warnung aus, wenn die Typen nicht zueinander kompatibel sind:
    \begin{center}\texttt{comparison of distinct pointer types lacks a cast}\end{center}\\
    Die Vorgaben bzgl. der Typkompatibilität bei Anwendung des Vergleichsoperators sind im Sprachstandard definiert, bspw. Kapitel 6.5.9 ANSI C99\footnote{
        final draft of C99 standard plus TC1, TC2, TC3); WG14; 2007, \url{https://www.open-std.org/jtc1/sc22/wg14/www/docs/n1256.pdf}, abgerufen 04.04.2025
    }: \blockquote[{6.5.9 Equality operators}]{
    ``One of the following shall hold [Anm.: ``Constraints``]: [\ldots]both operands are pointers to qualified or unqualified versions of compatible types; [\ldots]``
    }.\\
    Insgesamt ist die Typüberprüfung notwendig, weil der ternäre Operator hier durchaus implizite Typkonvertierung (beim Anwendung von \code{<}) durchführt, was zu unerwünschten Ergebnissen im weiteren Programmverlauf führen kann.\\
    So lässt ein Test des Makros ohne diese Codezeile bspw. Vergleiche von \code{float} mit \code{unsigned int} ohne weiteres zu.
\end{itemize}