\chapter{Aufgabe 2}

\section{Teil 1 und 2}

\textit{Was macht die Kernelfunktion \texttt{can\_request\_irq}?}\\

\noindent
Im Folgenden ist die Methode zur besseren Nachvollziehbarkeit angegeben.
Sie wurde aus dem Kernel-Code v2.6.38.8 aus der Datei \texttt{kernel/irq/manage.c}\footnote{
Quelle:\url{https://elixir.bootlin.com/linux/v2.6.38.8/source/kernel/irq/manage.c}, abgerufen 03.04.2025
} entnommen.\\

\begin{minted}[mathescape,
    numbersep=5pt,
    gobble=2,
    frame=none,
    framesep=2mm,
    fontsize=\small]{c}
    /*
     * Internal function that tells the architecture code whether a
     * particular irq has been exclusively allocated or is available
     * for driver use.
     */
    int can_request_irq(unsigned int irq, unsigned long irqflags)
    {
        struct irq_desc *desc = irq_to_desc(irq);
        struct irqaction *action;
        unsigned long flags;

        if (!desc)
            return 0;

        if (desc->status & IRQ_NOREQUEST)
            return 0;

        raw_spin_lock_irqsave(&desc->lock, flags);
        action = desc->action;
        if (action)
            if (irqflags & action->flags & IRQF_SHARED)
                action = NULL;

        raw_spin_unlock_irqrestore(&desc->lock, flags);

        return !action;
    }
\end{minted}

\noindent
Wie im Quellcode dokumentiert wird über die Methode ein \textbf{IRQ}\footnote{s.a.~\cite[52]{Man20d}} von einem Gerät/Gerätetreiber angefordert: Die Methode ermittelt zu dem übergebenen Argument \code{uint irq} mittels \texttt{irq\_get\_desc\_lock} einen Eintrag vom Typ \code{struct irq_desc} (s.a.~\cite[64 f.]{Man20d}) und setzt gleichzeitig eine Spinlock-Sperre\footnote{
    leichtgewichtige Sperre, realisiert durch \textit{busy-waiting}, vgl.~\cite[150]{Man20g}
} auf diese \code{struct} (\ldots \texttt{raw\_spin\_lock\_irqsave(&desc->lock, ...)}), damit keine nebenläufigen Änderungen an dieser erfolgen können.\\

\noindent
\code{irq_desc} beinhaltet u.a. einen Member, der eine gesetzte Sperre repräsentiert (\code{lock}) sowie einen Verweis \code{action} vom Typ \code{struct irqaction}, welcher Informationen zu dem mit dem IRQ assoziierten Gerät (\code{name}, \code{dev_id}, \ldots) enthält, und außerdem einen Member \code{flags}, eine Bitmaske, über die folgende Werte gesetzt werden können (Auszug\footnote{
Quelle: \url{https://elixir.bootlin.com/linux/v2.6.38.8/source/include/linux/interrupt.h#L101}, abgerufen 03.04.2025
}):

\begin{minted}[mathescape,
    numbersep=5pt,
    gobble=2,
    frame=none,
    framesep=2mm,
    fontsize=\small]{c}
    /*
     * These flags used only by the kernel as part of the
     * irq handling routines.
     *
     * IRQF_DISABLED - keep irqs disabled when calling the action handler.
     *                 DEPRECATED. This flag is a NOOP and scheduled to be removed
     * IRQF_SAMPLE_RANDOM - irq is used to feed the random generator
     * IRQF_SHARED - allow sharing the irq among several devices
     * IRQF_PROBE_SHARED - set by callers when they expect sharing mismatches to occur
     * IRQF_TIMER - Flag to mark this interrupt as timer interrupt
     * IRQF_PERCPU - Interrupt is per cpu
     * IRQF_NOBALANCING - Flag to exclude this interrupt from irq balancing
     * IRQF_IRQPOLL - Interrupt is used for polling (only the interrupt that is
     *                registered first in an shared interrupt is considered for
     *                performance reasons)
     * IRQF_ONESHOT - Interrupt is not reenabled after the hardirq handler finished.
     *                Used by threaded interrupts which need to keep the
     *                irq line disabled until the threaded handler has been run.
     * IRQF_NO_SUSPEND - Do not disable this IRQ during suspend
     *
     */
\end{minted}

\noindent
Interessant an der Stelle ist der Bitweise-Vergleich mit dem Flag \code{IRQF_SHARED}: Ist nämlich der IRQ bereits belegt (\code{desc->action}), muss dieser für eine erfolgreiche Anforderung von mehreren Treibern/Prozessen gleichzeitig geteilt werden können (\textit{Interrupt-Sharing},~\cite[59]{Man20d}).\\

\noindent
Nach der Überprüfung wird die Sperre auf die \code{irq_desc} wieder aufgehoben: Hat die Methode ermitteln können, dass der IRQ angefordert werden kann, wird \code{1} zurückgegeben, ansonsten \code{0}\footnote{
Anm.: C kennt im Original keinen Datentypen \textit{bool}, dieser wurde erst mit ANSI C99 eingeführt, vgl.~\cite[8]{ES3}
}- dies im Übrigen auch für den Fall, falls kein IRQ zu dem übermittelten Integer-Wert gefunden wurde.

\section{Teil 3}

\textit{\texttt{request\_region} ist ein Makro im Linuxkernel (Version 2.6.38.8). Welche Funktion wird in dem Makro eingesetzt? Diese Funktion verwendet
den kerneleigenen Datentyp \texttt{resource\_size\_t}. Welcher C-Datentyp
verbirgt sich dahinter?}\\


\noindent
\texttt{request\_region} wird in \texttt{source/include/linux/ioport.h} als Makro definiert.\\
In dem Makro wird die Funktion \texttt{\_\_request\_region} aus \texttt{source/kernel/resource.c} eingesetzt - tatsächlich an anderer Stelle auch einfach nur der Ausdruck \code{(1)}\footnote{
``We don't need no stinkin' I/O port allocation crap.``, \url{https://elixir.bootlin.com/linux/v2.6.38.8/source/arch/sparc/include/asm/floppy_32.h}, abgerufen 03.04.2025
} - wir konzentrieren uns auf ersteres:  \texttt{\_\_request\_region} fordert eine \textit{busy resource region} an - in diesem Kontext Speicherbereich von IO-Ports\footnote{
\texttt{struct resource ioport\_resource} (\url{https://elixir.bootlin.com/linux/v2.6.38.8/source/kernel/resource.c}, abgerufen 03.04.2025)
}).\\
Hierzu wird ein \texttt{resource\_lock} gesetzt (Schreib-/Lesezugriff), damit konkurrierende Prozesse nicht nebenläufig die benötigte Resource einplanen.
Im krtitischen Bereich wird die \textit{busy region} dann angefordert, der kritische Bereich wieder verlassen und die Sperre aufgehoben.

\begin{minted}[mathescape,
    numbersep=5pt,
    gobble=2,
    frame=none,
    framesep=2mm,
    fontsize=\small]{c}
    /**
     * __request_region - create a new busy resource region
     * @parent: parent resource descriptor
     * @start: resource start address
     * @n: resource region size
     * @name: reserving caller's ID string
     * @flags: IO resource flags
     */
    struct resource * __request_region(struct resource *parent,
    resource_size_t start, resource_size_t n,
    const char *name, int flags)
    {
        ...
        write_lock(&resource_lock);
        // kritischer Bereich
        ...
\end{minted}

\noindent
Hinter dem Datentyp \code{resource_size_t}\footnote{
\url{https://elixir.bootlin.com/linux/v2.6.38.8/source/include/linux/types.h}, abgerufen 03.04.2025
} steht der C-Datentyp \code{phys_addr_t}\footnote{\textit{ebd.}}, der - je nach Rechnerarchitektur - einer \textit{vorzeichenlosen Ganzzahl} (\texttt{u32} / \texttt{64}) der Wortgröße 32-Bit oder 64-Bit entspricht.

\begin{minted}[mathescape,
    numbersep=5pt,
    gobble=2,
    frame=none,
    framesep=2mm,
    fontsize=\small]{c}
    #ifdef CONFIG_PHYS_ADDR_T_64BIT
    typedef u64 phys_addr_t;
    #else
    typedef u32 phys_addr_t;
    #endif
\end{minted}

\noindent
Diese Typdefinition dient also zum Adressieren der (\textit{physikalischen}) Speicherbereiche, die von \texttt{\_\_request\_region} angefordert werden.